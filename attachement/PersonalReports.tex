\addcontentsline{toc}{section}{Personal Reports}
\section*{Personal Reports}

\subsection*{Thomas Kistler}
Nachdem die Studienarbeit letztes Semester sehr viel Spass gemacht hatte, freute ich mich sehr auf die Bachelorarbeit.
Da es sich um eine Fortsetzungsarbeit handelte, entfiel die Einarbeitung in ein neues Projekt.
Das fand ich fast etwas schade, da wir im Berufsleben wohl gerade dies als erstes meistern müssen.
Nichtsdestotrotz ist dem auch ein positiver Aspekt abzugewinnen:
Wir konnten direkt loslegen.\\

In der Studienarbeit wurden wir stark durch das CI aufgehalten und beeinträchtigt.
Diese Probleme konnten in der BA direkt zu Beginn aus dem Weg geräumt werden.
So konnten wir uns voll auf die Programmierarbeit stürzen.
Dies war uns sehr wichtig, denn wir wollten das CI als hilfreiche Unterstützung nutzen, statt es als notwendiges Übel zu betrachten.\\

Durch die intensive Auseinandersetzung mit dem Projekt kann ich sehr vieles für die Zukunft mitnehmen.
Neben den gesteigerten Programmierfähigkeiten konnte ich auch Kenntnisse in ganz anderen Bereichen wie z.B. LaTeX erwerben.
Im Vergleich zu der SA konnte ich die Fähigkeiten zur selbständigen Problemlösung weiter ausbauen.
Dies wird mir im späteren Berufsleben sicherlich zu Gute kommen.\\

Die BA hat mir auch die Wichtigkeit von Refactorings dargelegt.
Anfängliches Chaos konnte dank einer sauberen Analyse in organisierte Komponenten umstrukturiert werden.
Es war für mich sehr eindrücklich, wie sich dadurch die Wartbarkeit gesteigert hat.\\

Die Zusammenarbeit mit Marcel war super.
Einfache Aufgaben konnten wir gut aufteilen und so unsere Effizienz maximal halten.
Schwierige Aufgaben, oder Dinge bei denen wir beide viel lernen konnten, haben wir im Pair-Programming erledigt.
So konnten wir von uns gegenseitig profitieren und unsere Synergien perfekt nutzen.
Die Stimmung war immer locker und ich hätte mir keinen besseren Partner vorstellen können.
Die Corona Pandemie stellte hierbei überhaupt kein Hindernis dar.\\

Im Verlaufe der BA lernte ich das Projekt so gut kennen, dass es mir regelrecht ans Herz gewachsen ist.
Es war für mich deshalb persönlich wichtig, dass am Ende ein tolles Produkt abgeliefert werden konnte.
Umso mehr erfüllt es mich mit Stolz, dass das Plugin am Ende publiziert werden konnte.
Auch die Kollegen sind erstaunt, wenn man ihnen den Link zum Marketplace zusendet.
Dies ist umso erfreulicher, da es sich um eine Arbeit mit doch relativ hoher Komplexität gehandelt hat.\\

Ich werde die Bachelorarbeit als sehr positive Erfahrung in Erinnerung behalten.
Den weiteren Verlauf des Projektes werde ich sicherlich mitverfolgen.
Es würde mich sehr freuen, wenn der Language Server im Rahmen einer nachfolgenden Semester- oder Bachelorarbeit weiter vorangetrieben wird.