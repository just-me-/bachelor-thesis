\addcontentsline{toc}{section}{Personal Reports}
\section*{Personal Reports}

\subsection*{Thomas Kistler}
Nachdem die Studienarbeit letztes Semester sehr viel Spass gemacht hatte, freute ich mich sehr auf die Bachelorarbeit.
Da es sich um eine Fortsetzungsarbeit handelte, entfiel die Einarbeitung in ein neues Projekt.
Dies fand ich fast etwas schade, da wir im Berufsleben wohl gerade als erstes diese Herausforderung meistern müssen.
Nichtsdestotrotz hatte dies auch positive Aspekte:
Wir konnten direkt durchstarten.\\

In der Studienarbeit wurden wir stark durch das CI aufgehalten und beeinflusst.
Diese Probleme konnten in der BA direkt zu Beginn aus dem Weg geräumt werden, und so konnten wir uns voll auf die Programmierarbeit stürzen.
Im Verlaufe der BA lernte ich das Projekt so gut kennen, und habe so viel Zeit investiert, dass es mir richtig ans Herz gewachsen ist.
Es war für mich persönlich wichtig, dass am Ende ein tolles Produkt abgeliefert werden konnte.\\

Durch die intensive Auseinandersetzung mit dem Projekt kann ich sehr vieles für die Zukunft mitnehmen.
Neben den gesteigerten Programmierfähigkeiten konnten auch Kenntnisse in ganz anderen Bereichen wie z.B. LaTeX erworben werden.
Im Vergleich zu der SA konnte ich die Fähigkeiten zur selbständigen Problemlösung weiter ausbauen.
Dies wird mir im späteren Berufsleben sicherlich zu Gute kommen.\\

Die BA hat mir auch die Wichtigkeit von Refactorings dargelegt.
Anfängliches Chaos konnte dank einer sauberen Analyse in organisierte Komponenten eingeteilt werden.
Dies fand ich recht eindrücklich.\\

Die Zusammenarbeit mit Marcel war super.
Einfache Aufgaben konnten wir gut aufteilen und so unsere Effizienz maximal halten.
Schwierige Aufgaben, oder Dinge bei denen wir beide viel lernen konnten, haben wir im Pair-Programming erledigt.
So konnten wir von uns gegenseitig profitieren und unsere Synergien perfekt nutzen.
Die Stimmung war immer locker und ich hätte mir keinen besseren Partner vorstellen können.
Die Corona Pandemie stellte hierbei überhaupt kein Hindernis dar.\\

Es erfüllt mich mit Stolz, dass das Produkt am Ende publiziert werden konnte.
Auch die Kollegen sind erstaunt, wenn man ihnen den Link zum Marketplace schickt.
Dies ist umso erfreulicher, da es sich doch um eine Arbeit mit relativ hoher Komplexität gehandelt hat.\\

Ich werde die Bachelorarbeit als sehr positiv in Erinnerung behalten.
Den weiteren Verlauf des Projektes werde ich sicherlich mitverfolgen.
Ich würde mich sehr freuen, wenn das Projekt im Rahmen einer nachfolgenden Semester- oder Bachelorarbeit weiter vorangetrieben wird.


