\section{Abstract}
\label{section:abstract}

 \intnote{Stand aktuell: 1998 / 2000 Zeichen}
 \intnote{mit 3 kleinen bildern könnte man 3400 zeichen, aber die kurze version gefällt mir eig eh besser.}

\textbf{Initial Situation}\\
Dafny is a formal programming language to proof a program's correctness.
In a preceding bachelor thesis, a plugin for Visual Studio Code had been created to access Dafny-specific static analysis features.
The plugin communicates with a language server, using Microsoft's language server protocol, which standardizes communication between an integrated development environment (IDE) and a language server.
The language server itself used to access the Dafny library, which features the backend of the Dafny language analysis, through a proprietary JSON-interface.
In a preceding term project, the language server was integrated into the Dafny backend to render the JSON-interface obsolete.\\

\textbf{Objective}\\
This bachelor thesis continues the preceding term project and contains two major goals.
Firstly, the previously implemented prototype has to be improved in usability, stability and reliability.
Secondly, a symbol table has to be implemented to facilitate the development of symbol-oriented functionality like rename or auto completion.
The symbol table is supposed to open a wide range for further development.\\

\textbf{Result}\\
A symbol table was created using the visitor pattern.
Every symbol contains information about its parent, its children and its declaration.
This keeps navigation within the symbol hierarchy very simple.
The implemented features go to definition, rename, code lens, hover information and auto completion benefit by the symbol table and are no longer required to contain complex algorithms.
The offerings of the symbol table can be accessed for future extensions, such as auto formatting or code highlighting.\\

Aside features based on the symbol table, pre-existing functionality was revisited as well.
The prototype achieved a quality to now be ready for distribution as a preview version on the Visual Studio Code market place.
