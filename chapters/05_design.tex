\section{Design}
\label{chapter:design}
This chapter contains discussions of fundamental design decisions that were made.
The first section contains reasoning about the choice of technologies, such as programming languages.
Afterwards, architectural discussions follow, separated by client and server.

\subsection{Technologies}
The choice of technologies is determined by external circumstances.
The client is written in TypeScript, since this is required for VSCode plugins \cite{vscodeAPI}.
To work on the client, Visual Studio Code is a natural choice, since it contains an engine to directly launch and test the coded plugin.
The server is written in \CsharpWithSpace using .NET Framework 4.6.1, since Dafny is already coded in this
configuration\footnote{During the course of this project, Dafny was updated to .NET Framework 4.8} and direct integration is a key goal of the project.
To work on the server, Visual Studio 2019 is used with the ReSharper extension.\\
The CI will be realized with GitLab, since HSR offers it for free and the CI-pipeline of the prototype was pre-existing.

\subsection{Client}
The client consists of the VSCode plugin written in TypeScript.
It establishes a connection to the server using the language server protocol.\\

The client is supposed to be very simple and only responsible for UI tasks, while as much logic as possible should be implemented at the server side.
This allows to implement support for other IDE's with as little effort as possible.\\

In the following, it is discussed how this lightweight design of the client will be achieved.
Furthermore, the separation into components is also explained.

\subsubsection{Initial Situation}
The original client was already refactored within the prototype.
A lot of dead code was removed and logic was moved to the server.
Figure \ref{fig:client_then} gives an impression of the architecture at the beginning of this project.

\begin{figure}[H]
    \centering
    \includegraphics[width=10cm]{client_then}
    \caption{Client Architecture - Term Project}
    \label{fig:client_then}
\end{figure}

In this simplified representation, the client architecture appears very tidy.
However, the individual components were very large.
Almost all members were public.
This led to high coupling and low cohesion.
Furthermore, there were various helper classes which were not grouped into sub-packages.
This made it challenging to maintain the code.
Furthermore, it was difficult to identify all dead code passages due to the non-transparent dependencies.\\

Because of those problems it was decided to redesign the client from scratch.

\subsubsection{New Architecture}
To achieve the goal of a more manageable architecture and to reduce coupling, the following measures were targeted:
\begin{itemize}
    \item As a first step, all areas of responsibility were divided into separate components.
    \item The components were then grouped into packages as you can see in figure \ref{fig:client_now_packages}.
\end{itemize}
The new distribution into these packages is discussed in the following sections.\\

\begin{figure}[H]
    \centering
    \includegraphics[width=6cm]{client_now_packages}
    \caption{Client Architecture - Packages}
    \label{fig:client_now_packages}
\end{figure}

Additionally to the measures mentioned above, all logic was detached from the extension class, which forms the main component.
This resulted in the root directory containing only a lightweight program entry point.
The rest of the logic was split between the created packages.\\

As a little extra, each component contains code documentation to help other developers to get started quickly.
This is also helpful because they are displayed as hover tooltips.\\

{\bf \code{extension}} \textendash{}
This component is the aforementioned \code{main}-component of the plugin and serves as an entry point.
The incorporated code has been minimized.
Only one server instance is started.
The logic is located entirely in the server package.\\

{\bf \code{server}} \textendash{}
The server package contains the initialization of the language server and the establishment of the connection between the client (plugin for VSCode) and the Dafny language server.
In addition, all server requests, which extend the LSP by custom functionality, are sent to the server via this package.\\

{\bf \code{typeInterfaces}} \textendash{}
In the new architecture, no \code{any} type should be used anymore.
All types, in particular types created specifically for custom requests such as \code{CounterExampleResults}, were defined by interfaces. \\

{\bf \code{ui}} \textendash{}
The UI is responsible for all visual tasks, especially VSCode commands and context menu additions.
Core components like the status bar are also included in this package.\\

{\bf \code{localExecutionHelper}} \textendash{}
This package contains small logic extensions like the execution of compiled Dafny files.
The UI package accesses this package.\\

{\bf \code{stringResources}} \textendash{}
All string resources and command identifiers are defined in this package.
It is used by the \code{ui} package. \\

In the following sections, the individual components and their contents are described in more detail.

\subsubsection{Components}
Figure \ref{fig:client_now_classes} shows a more detailed view of the client, including the components within the packages.
The contents of \code{typeInterfaces} and \code{stringResources} have been omitted for clarity.\\

\begin{figure}[H]
    \centering
    \includegraphics[width=8cm]{client_now_classes}
    \caption{Client Architecture - Components}
    \label{fig:client_now_classes}
\end{figure}

It can be seen that only \code{compile} and \code{counterExample} exist as dedicated server access classes.
All other features, such as CodeLens or AutoCompletion are natively supported by the LSP.
This means, that no additional client logic is necessary to support these features.
Since Compile and CounterExample are custom requests,
their handling has to be implemented manually within the client.\\

If VSCode receives a AutoCompletion response, the LSP standard defines how VSCode has to handle it -
namely displaying the completion suggestions and insert the text which the user selects.
This server-side-oriented implementation via the LSP is a great enrichment to keep the client lightweight.
Adaption to other IDE's is very simple this way.

\begin{figure}[H]
    \centering
    \includegraphics[width=\textwidth]{client_now_methods}
    \caption{Client Architecture - Components and Public Methods}
    \label{fig:client_now_methods}
\end{figure}

Figure \ref{fig:client_now_methods} shows the public methods of each component.
Only these are accessible.
Instance variables were set to private visibility.
Constructors were not included for simplicity.
The contents of type interfaces and string resources were also omitted for clarity.\\

This distribution has certain upward dependencies, which is not optimal.
The \code{ui} package accesses the server package and part of the server package accesses the \code{ui} package.
Nevertheless, we have decided on this grouping,
so that the server access functionality is encapsulated.

\subsubsection{Logic}
The logic contained in the client has been reduced to a minimum.
This has the advantage, that porting the client to other IDEs is as easy as possible.
This section describes where and why the client still contains logic.\\

{\bf Server Connection} \textendash{}
Handling of the connection to the language server and sending API requests.
In addition, the client has a simple logic that certain server requests (such as updating the CounterExample)
are sent at most twice per second. \\

{\bf Execute Compiled Dafny Files} \textendash{}
The execution of compiled Dafny files is relatively simple.
One distinguishes whether the execution of \code{.exe} files should be done with mono (on macOS and Linux operating systems) or not. \\

{\bf Notifications} \textendash{}
The client is able to receive notification messages from the server.
These notifications are split into three severity levels:
\begin{itemize}
    \item Information
    \item Warning
    \item Error
\end{itemize}
The corresponding logic in the client receives these messages and calls the VSCode API to display a popup message. \\

{\bf Commands} \textendash{}
To enable the user to actively use features (such as compile),
the corresponding method calls must be linked to the VSCode UI.
There are three primary links for this:
\begin{itemize}
    \item Supplementing the context menu (right-click).
    \item Keyboard shortcuts.
    \item Entering commands via the VSCode command line.
\end{itemize}

{\bf Status Bar} \textendash{}
The information content for the status bar is delivered entirely by the server.
The client only takes care of the presentation.
Therefore, certain event listeners must be registered, which react to the server requests.
Furthermore, the received information is buffered for each Dafny file.
This allows the user to switch seamlessly between Dafny files in VSCode
without having the server to send the status bar information
(like the number of errors in the Dafny file) each time.\\

{\bf CounterExample} \textendash{}
Similarly to the status bar, the \code{CounterExample} component will have a buffer.
For each Dafny file in the workspace, a buffer stores whether the user wants to see the CounterExample or not.
This way the CounterExample is hidden when the user switches to another file
and automatically shown again when switching back to the original Dafny file.

\subsubsection{Types in TypeScript}
As already mentioned in the previous sections, \code{any}-types were largely supplemented by dedicated type interfaces.
This prevents type changes of variables as known in pure JavaScript.
Typed code is accordingly less error-prone - especially for unconscious type castings. \\

There are individual built-in datatypes like \code{number}, \code{boolean} and \code{string} \cite{ts-types}.
For custom types, such as compilation results, we have defined separate interfaces.
An example is shown in listing \ref{lst:typeinterface}.

\begin{lstlisting}[language=typescript, caption={Type Interface Supplementing \code{any}-types}, captionpos=b, label={lst:typeinterface}]
export interface ICompilerResult {
    error: boolean;
    message?: string;
    executable?: boolean;
}
\end{lstlisting}

%%%%%%%%%%%%%%%%%%%%%%%%%%%%%%%%%%%%%%%%%%%%%%%%%%%%%%%%%%%%%%%%%%%%%%%%%%%%%%%%%%%%%



\subsection{Server}
This subsection documents the architectural layout of the server.
The server consists of eight different packages:
\begin{itemize}
    \item \code{Main}: Starting of the language server.
    \item \code{Handler}: Components handling LSP requests.
    \item \code{Core}: Contains provider-classes that supply the handlers with the necessary information.
    \item \code{CustomDTOs}: Contains all custom parameters and results like \code{CompilationResults}.
    \item \code{WorkspaceManager}: Represents the user workspace with its buffered files.
    \item \code{SymbolTable}: Handles anything related to the symbol table (creation, navigation, access).
    \item \code{Tools}: Individual, isolated services needed across the project.
    \item \code{Commons}: Classes accessed from multiple components, such as the language server configuration.
    \item \code{Resources}: Strings and filesystem resources.
\end{itemize}

Figure \ref{fig:server_overview} shows a broad overview of the server layering.
The packages are discussed in a detailed matter below from top to bottom.

\begin{figure}[H]
    \centering
    \includegraphics[width=11cm]{design/server1_overview.png}
    \caption{Server Overview}
    \label{fig:server_overview}
\end{figure}

\subsubsection{Main Component And Handlers}
The \code{Main} component is responsible to start the language server.
The capabilities of the server can be registered by injecting \textit{Handlers.}
The handlers themselves will have to implement a specific interface by OmniSharp, for example \code{IRenameHandler}.
These interfaces demand a \code{Handle} method with the proper arguments and return values to answer the request.\\

Figure \ref{fig:server_basic_idea} shows the basic, fundamental layout of the server architecture.
Be aware that this illustration is highly simplified for a better understanding.
Only one handler is shown.\\

\begin{figure}[H]
    \centering
    \includegraphics[width=12.5cm]{design/server2_basic_idea.png}
    \caption{Basic Server Concept}
    \label{fig:server_basic_idea}
\end{figure}

The handlers contain some boilerplate code for registering the capability and keeping references to the language server.
To keep classes concise, the \code{Handle} methods will not calculate the result themselves, but forward to call to a \code{Provider} from the \code{Core} package.
This component is then responsible to provide the proper result.
This way, the core logic can be easily tested since it is in a separate component.\\

\subsubsection {Core}
The core package contains the actual core logic for the features.
There is one class for each feature, for example \code{RenameProvider} or \code{DiagnsoticsProvider}.
The provider is invoked by the handler and responsible to calculate the result, that has to be sent back to the client.
For that, the provider will invoke the \code{WorkspaceManger} and/or the \code{SymbolTableManager} and operate on these.
Figure \ref{fig:server_core} shows the core package with the example of the \code{RenameProvider}.
Note that the symbol table is injected using the interface \code{ISymbolTableManager}.
By just creating a fake instance, the providers can very easily be tested using this architecture.

 \intnote{and/or? was den nun und wann ;) and oder or nehmen, das verwirrt sonst}

\begin{figure}[H]
    \centering
    \includegraphics[width=12.5cm]{design/server6_core.png}
    \caption{Core Package Excerpt with RenameProvider}
    \label{fig:server_core}
\end{figure}

\subsubsection{CustomDTOs}
To provide the results of custom LSP requests, result wrapper classes had to be created.
Since both, \code{Handlers} and \code{Providers} require these, they were extracted to a separate package.
It just contains the wrapper classes for
\begin{itemize}
    \item \code{CounterExampleResults}
    \item \code{CounterExampleParams}
    \item \code{CompilationResults}
    \item \code{CompilationParams}
\end{itemize}

\subsubsection{Workspace Manager}
A Dafny project consists out of multiple \code{.dfy} files.
Thus, it is evident to create a class \code{PhysicalFile}.
It just has two properties, a file path and the file content.
Since the user is manipulating files by editing the code, it also provides an \code{Update} method.\\

To be able to provide all the necessary functionality, there is more information associated with a single file than just the content:
\begin{itemize}
    \item Is the file valid?
    \item Does it contain errors?
    \item What does the compiled Dafny und Boogie program look like?
    \item What symbols occur in the file?
\end{itemize}

Thus, a class \code{FileRepository} will be created.
It contains a \code{PhysicalFile}, but also all of the requested information in the list above.
For that, a wrapper \code{TranslationResults} will be created, containing information about errors and compiled items.
Finally, to have a link to the symbol table, a reference to a \code{SymbolTableManager} will be added.\\

Since \code{TranslationResults} and \code{PhysicalFile} are used in multiple parts of the code, these classes will be place within a \code{Commons} package.
Cyclic dependencies can be avoided this way.\\

To request information about a file, or to update a file, a component \code{WorkspaceManager} will be created.
It contains a dictionary, linking a file identity (as URI) to a \code{FileRepository}.
It offers a method to update a file, but also to get information about a file.\\

\begin{figure}[ht]
    \centering
    \includegraphics[width=10cm]{design/server3_workspace.png}
    \caption{Workspace Component}
    \label{fig:server_workspace}
\end{figure}

Whenever a change is triggered, the \code{Update} method of the \code{WorkspaceManager} can be called to apply them.
The manager will forward the call towards the \code{FileRepository}, which will adjust the \code{PhysicalFile},
but also generate new \code{TranslationResults} by invoking the Dafny Access package.
Last but not least, the symbol table is also recalculated invoking the proper component discussed later.
Any changes are then stored in its \code{Buffer} property.\\

\subsubsection{Dafny Access}
This component is responsible to access Dafny's backend.
The core class in this package is the \linebreak \code{DafnyTranslationUnit}.
It receives a \code{PhysicalFile} in the constructor.
The only public method \code{Verify} will start the Dafny verification process for that file.
As a result, the process will yield a list of errors, warnings and information objects, but also a precompiled \textit{dafnyProgram} and a precompiled \textit{boogieProgram} for later reuse.
All of this information is stored in the \code{TranslationResults} wrapper class and returned as a result.\\

\begin{figure}[H]
    \centering
    \includegraphics[width=10cm]{design/server4_dtu.png}
    \caption{Dafny Translation Unit}
    \label{fig:server_dtu}
\end{figure}


\subsubsection{SymbolTableManager}
All tasks related to the symbol table are moved into a dedicated package.
The symbol table is built based on a precompiled \code{dafnyProgram}.
It is available from the \code{WorkspaceManager} for every file and can be reused by this package.\\

The symbol table manager contains of five classes:
\begin{itemize}
    \item \code{SymbolInformation}, a class containing just raw data about a symbol.
    \item \code{SymbolUtilities}, a static class allowing operations on single symbols.
    \item \code{Navigator}, a class to navigate through the symbol table.
    \item \code{Generator}, the class that builds the symbol table.
    \item \code{Manager}, a class to provide easy access to the symbol table for outside users.
\end{itemize}

The class \code{SymbolInformation} is supposed to contain any necessary data about every symbol.
\begin{itemize}
    \item In which file is it?
    \item On what line, on what column?
    \item What is it's parent, what children are in the body of the symbol?
\end{itemize}

The class \code{Navigator} is responsible to navigate through the symbol table.
Since every symbol is supposed to know about its parent and children, it is legit to speak about a symbol tree.
The \code{Navigator} will move along the tree, for example to locate the symbol at a certain position.\\

The \code{Generator} is responsible to build the table.
It will already make use of the navigator, namely to locate declarations of symbols.
The result will be a root symbol, to which everything else is attached.\\

The \code{SymbolUtilities}-class allows logical operations on symbols.
These were isolated for abetter testability.\\

The \code{Manager} is constructed once a symbol table has been completely built.
It stores the root symbol and invokes the navigator to provide easy access to the symbol table construct for the outside user.\\

All components are programmed against an interface, so that fakes can be created for testing.\\

\subsubsection{Resources}
String resources, such as error messages, are extracted to a specific resource package.
This way, they are easy to maintain and adjustable at a central place.
A similar system of centralization of string resources is followed as with the client.

\begin{figure}[H]
    \centering
    \includegraphics[width=6cm]{design/server7_resources.png}
    \caption{String Resources}
    \label{fig:server_resources}
\end{figure}

As seen in figure \ref{fig:server_resources}, they are further split into sub-categories to keep them organized.

\subsubsection{Tools}
Classes that take care of general tasks are put into the \code{Tools} package.
For example, one class in this package is responsible for delivering information about the folder structure, so that relative paths can be resolved.
Another class is responsible to create a proper logger, which can then be used throughout the language server.
All these auxiliary classes were collected within the \code{Tools} package.\\

\textbf{Tools.ConfigInitializer}\\
The setup of the language server settings is using five different classes.
To keep packages concise, a sub-package was created just for config initialization.
The classes are:
\begin{itemize}
    \item One class to parse launch arguments.
    \item One class to parse the config file.
    \item One class to report errors occurring during the process.
    \item One class to represent errors.
    \item One class coordinating the whole process and invoking all other classes.
\end{itemize}
Since the final config setting are accessed from multiple stages inside the code, the actual language server config is located inside the \code{Commons} package.
It would be the sixth class related to the config initialization.

\begin{figure}[H]
    \centering
    \includegraphics[width=10cm]{design/server7_config.png}
    \caption{Configuration Initialization Layout.}
    \label{fig:server_config}
\end{figure}

The following parameters are configurable:
\begin{itemize}
    \item Output Stream File Location
    \item Logfile Location
    \item Minimum Loglevel
    \item LSP Transmission Mode (Full / Incremental)
\end{itemize}
 \intnote{trennung von config und ressources aufzeigen.}


\subsubsection{Workflow Overview}
%Sourcecode des flowgraphen:
%Language Server->Handler:textdocument/rename
%Handler->Provider:Forwarding request
%Provider->WorkspaceManager:Requesting FileRepository
%WorkspaceManager->Provider:Delivering FileRepository
%Provider->Provider:Calculate Result
%Provider->Handler:Deliver Result
%Handler->Language Server: Forward Result
For a regular request, the language server calls the proper handler to process the request.
The handler will then retrieve the \code{FileRepository} from the \code{WorkspaceManager} and extract the necessary information.
This information is forwarded to the provider, which calculates the result and returns it.
This is illustrated in sequence diagram \ref{fig:workflow1}.

\begin{figure}[H]
    \centering
    \includegraphics[width=\textwidth]{design/workflow1_regular.png}
    \caption{Seqience Overview of a Rename Request}
    \label{fig:workflow1}
\end{figure}

However, if an update is triggered, the workflow is slightly different.
The handler will now actually request the \code{WorkspaceManager} to update a file, which will trigger the whole verification process and recalculate the symbol table, if possible.
At the end, the handler retrieves the updated \code{FileRepository}.
Now, it forwards the repository to the \code{VerificationProvider}, which extracts, assembles and sends diagnostics to the client.
This is shown in \ref{fig:workflow2}

\begin{figure}[H]
    \centering
    \includegraphics[width=\textwidth]{design/workflow2_update.png}
    \caption{Updating a File After a File has Been Changed}
    \label{fig:workflow2}
\end{figure}


\subsection{Symbol Table}
Since the symbol table is a core concept of this thesis, it will explained further in this section.
In the following, the design of the symbol table is described as well as it is built up with the help of the visitor pattern.

\subsubsection{Symbol Table Design}
To provide the necessary information for features like Rename, GoToDefinition or AutoCompletion,
the symbol table needs to be very versatile and it must be able to handle the following challenges:
\begin{itemize}
    \item Find a symbol at the cursor position and go to symbol declarations for the GoToDefinition feature.
    \item Find all occurrences of a symbol for the Rename feature.
    \item Get all available symbols within a scope for the AutoCompletion feature.
    \item Know about usages of a symbol for the CodeLens feature.
\end{itemize}

\textbf{Find symbol at cursor}\\
To overcome this task, each symbol knows about its position.
The end of the symbol can be deduced by adding the identifier-length to the start position.
A dedicated logic can then decide if the cursor is wrapped by the symbol or not.\\

\textbf{Find available symbols in scope}\\
For this task, information about scopes need to be available.
The mentioned positional data is thus extended by a \code{BodyStartToken} and a \code{BodyEndToken}.
Of course, only symbols with an actual body will have these properties populated.\\

To find all available symbols, every scope needs to know, what is declared inside it.
Thus, a property \code{Children} is necessary.
It is implemented as a hash list for instant access.
One further has to divide between declarations and any occurring symbols.
Another property \code{Descendants} contains any occurring symbol in a scope, not only declarations.\\

\textbf{Find declaration}\\
For this task, each symbol has to hold a reference to the declaration.
To be able to locate the declaration beforehand, it must be possible to move to the parent scope and scan it for declarations.
Aside the already mentioned \code{Children}-property, a \code{Parent}  property is necessary. \\

\textbf{Find usages of symbol}\\
A list of usages need to be provided, containing a reference to every usage of the symbol. \\

All of this leads in the following structure of the class \code{SymbolInformation} shown in figure \ref{fig:symboltabledesign}.

\begin{figure}[H]
    \centering
    \includegraphics[width=6.5cm]{design/SYMBOLTABLE.png}
    \caption{Symbol Table Design}
    \label{fig:symboltabledesign}
\end{figure}

\subsubsection{Visitor Pattern}
To generate a symbol table, it is common to use the visitor programming language pattern by Gamma et al \cite{gofBook}. \glsadd{visitor}
The pattern is used to navigate through, mostly tree-based, data structures and execute operations while doing so.
The goal of the pattern is to separate the navigation through the data structure, and the operations that take place when visiting.\\

Consider any tree based data structure.
Every node in the tree is supposed to offer an \code{Accept(Visitor v)} method as shown in listing \ref{lst:accept}.
This method will accept the visitor, this is, it will execute the visitor's operation on the node itself.
Further, it will also call the \code{accept}-methods of its child nodes.
Thus, a typical implementation of an acceptor would look like this:

\begin{lstlisting}[language=csharp, caption={Example for Accept}, captionpos=b, label={lst:accept}]
public void Accept(Visitor v) {
    v.Visit(this);
    foreach (Node child in this.Children) {
        child.Accept(v);
    }
}
\end{lstlisting}

Note that the navigational aspect - the \code{foreach} loop - is inside the accept method, but nothing is told about the visit operation.
The visitor can do whatever it wants with the node, for example print it to the console.
To work with every node that may occur, the visitor must overload the \code{Visit(Node n)} method for each possible subclass of \code{Node}.
Within a tree, this usually just concerns nodes and leaves.
For a symbol table, possible node types are any kind of expressions and statements.\\

A visitor implementation could look like shown in listing \ref{lst:visitor}.

\begin{lstlisting}[language=csharp, caption={Example for Visitor}, captionpos=b, label={lst:visitor}]
public class Printer : Visitor {
    public override void Visit(Node n) {
        Console.WriteLine("Node: " + n.ToString());
    }
    public override void Visit(Leaf n) {
        Console.WriteLine("Leaf: " + n.ToString());
    }
}
\end{lstlisting}

\subsubsection{Global Symbol Table vs Symbol Table per File}
\label{section:globalSymbolTable}
The current design creates a symbol table per file, and attaches it to the file repository.
This variant was chosen for simplicity.
However, it is technically not correct.
A better variant would be, to use a global symbol for the entire workspace.
This has two reasons:
\begin{itemize}
    \item If a file gets included elsewehre - the file is not aware about this. CodeLens and Rename will not work properly.
    \item AutoCompletion may only suggest symbols in the current file. Symbols available in other files are not known.
\end{itemize}

Since the managment of a global symbol table is not as easy, it was decided to go with a symbol table per file in a first step.
A global symbol table is a possible extension point for the feature.
The procedure is, if a file gets updated, to locate the current position of the pre-existing symbol table, detach the old one and attach the new one.
While this is not difficult, handling of global namespaces that do not know each other may suddenly become quite tricky,
since there is no longer just 'one' global space, but one per file, as long as they do not include each other.\\

A global symbol table would also increase performance and especially resource usage, since if a library is included by multiple files, it's symbol table does not have to be built multiple times.\\

An alternative solution to make features like Rename and GoToDefinition would be to keep a table, which file includes which other files.
If a file A is included in file B, an additional buffer within file A indicates that it was included by file B.
Keeping all these references up to date is also connected with logical effort.
Therefore, it is advisable to implement the global and updateable symbol table instead.
