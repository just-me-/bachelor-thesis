\section{Implementation}

\subsection{Symbol Table}

\subsubsection{Feature Support}
Since we have object information (and not just strings anymore) with our self-written symbol table,
the whole position to string parsing was dropped. \\

In our old version we had to find out from the current cursor position which word in the code could be meant.
Then we iterated over the whole symbol table and checked if there was a symbol with the same string as name.
The first match was looked at as a meant symbol. \\

Our new design eliminates all of this effort and avoidable assumptions.
We access the currently marked symbol directly via the position data.
String comparisons and corresponding string extractions are completely eliminated.
This leads to better performance and above all to reliable symbol references.

\subsubsection{Code Review}
Fabians Feedback aus der SA... neues Review. "Tu Gutes und sprich davon".

\subsection{Client Code Review}
Fedback Fabian und Thomas. Interface für Koopelung, weniger Kommentar, mehr Interfaces. Besseres Naming für Variablen. Mehr Interfaces.
Beschreiben wie es nun neu aussehen wird.

\subsection{Mono Support for macOS and Linux}
Eines der Kernzeiele war es, Support für mehrere Plattformen zu bieten. Dh nebst Windows auch macOS und Linux.
Da wir in unserer SA von Core auf Framework umsteigen musste, stand fest, dass wir mono für den Support auf Linux und macOS brauchen.
(warum in der SA; plficht wegen dafny core. was ist mono)

Leider funktionierts nicht.
Anssätze die wir probiert haben. verschiedene mono versionen, angefragt im slack. antwort erhalten?
github issues: allgemein probleme mit lunux/mac weil primär auf windows und gar nicht auf mac getestet wird. (heikle aussage selbs tmit quelle) 

\cite{sa}
\cite{mono-slack}
\cite{mono-git}
