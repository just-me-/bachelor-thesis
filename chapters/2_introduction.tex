\section{Management Summary and Introduction}
\label{section:introduction}
In this chapter, Dafny and the language server protocol (LSP) are explained to provide the reader with the necessary context.
Afterwards, the motivation and the goals of the thesis are stated in more detail.

\subsection{Dafny}
\label{section:introduction:dafny}
Dafny is a compiled language that targets \Csharp which is optimized to prove formal correctness \cite{dafnyWiki}.
Dafny bases on the language \textit{Boogie}, which uses the \textit{Z3} automated theorem prover for discharging proof obligations \cite{dafnyWiki}.
That means, a programmer can define a precondition - a fact that is just given at the start of the code.
The postcondition on the other hand is a statement that must be true after the code has been executed.
The postcondition is also defined by the programmer.
In other words, under a given premise, the code will manipulate data only thus far, so that also the postcondition will be satisfied.
Dafny will formally proof this.
If it is not guaranteed that the postcondition holds, an error is stated.\\

The following code snippet shows an example.
The value \code{a} is given, but it is required to be positive.
This is the precondition.
In the method body, the variable \code{b} is assigned the negative of \code{a}.
Thus, we ensure, that \code{b} must be negative, which is the postcondition.
\begin{lstlisting}[language=dafny, caption={Simple Dafny Example}, captionpos=b, label={lst:simpleDafnyExample}]
method Negate(a: int) returns (b: int)
requires a > 0
ensures b < 0
{
    b := -a;
}
\end{lstlisting}
This example is of course trivial.
In a real project, correctness is not that obvious.
With Dafny, a programmer can be sure if the program is correct.
Since the proof is done with formal, mathematical methods, the correctness is guaranteed.\\

If Dafny is unable to perform a proof, the user can assist by creating lemmas.
Lemmas are mathematical statements.
For example, a lemma could be that a factorial number is never zero.
If we define a simple function \code{Factorial}, and afterwards divide through the result of \code{Factorial}, Dafny will state that this might be a division by zero error.
But if we assert that a factorial number can never be zero, verification can be completed successfully.
\begin{lstlisting}[language=dafny, caption={Lemmas}, captionpos=b, label={lst:lemma}]

function Factorial(n: nat): nat
{
    if n == 0 then 1 else n * Fact(n-1)
}

lemma FactorialIsPositive(n: nat)
ensures Fact(n) != 0
{}

function Foo(n: nat): float
{
    FactorialIsPositive(n);
    100 / Fact(n)
}
\end{lstlisting}
\intnote{ich würde den absatz streichen, was meinst du?}

\subsection{Language Server Protocol}
\intnote{abschnitt ist scheisse, kannst du hier was schreibne? bekomms nich hin. find ich aber als core technoplogy gehör tdas hier hin. hab asuserdem das gbild von chapter 3 geklaut :()}
The language server protocol (LSP) was created to unify communication between an integrated development environment (IDE) and a language server.
The language server is responsible to offer support for a specific programming language.
Offering support means providing analysis features such as code verification, but also refactorings like renaming or auto formatting.\\

Instead of every IDE providing individual support for each programming language, an IDE supporting the LSP can simply connect to the language server, if its available.
The language server will then handle any requests it supports and it seems the IDE offers support for the programming language.

\begin{figure}[H]
    \centering
    \includegraphics[width=\textwidth]{03_lspcommunication.png}
    \caption{Communication Benefit of LSP}
    \label{fig:lspcommunication}
\end{figure}

\subsection{Initial Solution}
\label{section:introduction:initialsolution}
In a previous bachelor thesis by Markus Schaden and Rafael Krucker, a plugin for Visual Studio Code was created to support Dafny \cite{ba}.
The plugin was particularly appreciated by the "HSR Correctness Lab" \cite{correctnessLab} to make coding in Dafny easier.
The pre-existing solution offered a LSP-client for Visual Studio Code, which connected to a language server.
The language server was written in TypeScript, since the plugin was also created using TypeScript.
To communicate with the Dafny backend, the language server used a proprietary JSON-interface.
Information provided by Dafny was parsed from the console output.
Thus, functionality was limited to what Dafny printed onto the console.\\
In the preceding semester project \cite{sa}, the language server was integrated into the Dafny backend and rewritten in \Csharp.
Thus, any functionality was directly accessible and the proprietary JSON-interface, as well as console parsing could be omitted.
All features had to be reimplemented to satisfy the new architectural layout.

\begin{figure}[h]
    \centering
    \includegraphics[width=\textwidth]{02_oldBAvsSA.png}
    \caption{Architecture before (left) and after (right) the preceding term project}
    \label{fig:oldBAvsSA}
\end{figure}


\subsection{Goals}
In this bachelor thesis, the goal was to further improve the language server and the VSCode plugin, which were created in the preceding term project.

In the first segment of the project, the pre-existing features that do not depend on the symbol table had to be improved.
Also, any leftovers of the preceding term project had to be completed.
This included:
\begin{itemize}
    \item CI: Install quality measures
    \item CI: Implement integration tests
    \item Counter Example: Simplify representation
    \item Verification: Use a workspace buffer to store the verification results
    \item Verification: Display warnings as well, not only errors.
    \item Compilation: Finish Dafny integration, use buffered verification results
\end{itemize}

Afterwards, features that profit by the symbol table were targeted.
These features primarily include
\begin{itemize}
    \item goto definition
    \item auto completion
    \item code lens
    \item renaming
\end{itemize}
During Dafny's compilation process, an abstract syntax tree (AST) is generated, but that syntax tree does not contain all information required for the desired functionality.
For example:
\begin{itemize}
    \item to go to a symbol's definition, every name segment should know about its declaration
    \item to provide proper auto complete suggestions, all available declarations in a scope have to be known
    \item to rename a symbol, all occurrences of a symbol must be stored
    \item to display code lens, all usages of a declaration must be noted
\end{itemize}
Thus, a dedicated symbol table containing all of this information had to be implemented.\\

By offering the symbol table, the development of further features is significantly facilitated.
For example, the symbol table should provide enough information, that the depth of scope can be deduced.
This allows the creation an auto-formatting feature, which indents user code according to the code depth.\\

Aside the changes in the language server, the VSCode plugin was further refactored to be as lightweight as possible.
This allows easy adaption to other IDE's, since any logical calculations are done inside the language server.




\subsection{Results}
\intnote{Trennung von Features, Verbesserungen für Entwicklern, Verbesserung über Symboltable?}\\

Within this bachelor thesis, the development of the pre-existing language server and its VSCode client was continued.
Significant improvements could be achieved.

The client is now lightweight and contains almost no logic anymore.
This makes the adaption to other IDE's very simple.
The server is now able to create a symbol table containing any information required for the langue analysis features.
Pre-existing features and algorithms were improved to gain more reliability and a better user experience.

The following features are supported by the server:
\begin{itemize}
    \item Syntax highlighting
    \item Verification, highlighting of errors
    \item Compilation
    \item Show Counter Example
    \item Code Lens
    \item Auto Completion
    \item Renaming
    \item Hover Information
\end{itemize}




\intnote{das hier unten, braucht es das alles eigentlich? is doch schon viel zu technisch. find ich auch net gut.}
The dafny verification process follows now a clear strucutre.
First, the Dafny lexer is called.
Then, the Dafny resolver performs semantic checks.
Afterwards, the Dafny program is translated into Boogie programs which are then logically verified.
Any errors during this process are collected and properly displayed to the user.
Intermediary compilation results are stored for later reuse.
Previously, only logical errors were displayed.
Warnings or Informations were not displayed at all.

Compilation will use the precompiled result and is much faster by now.
The user can easily enter custom compilation arguments within the Visual Studio Code client.
The representation of counter examples is now less cryptic and easier to read.\\

By using the visitor pattern, the Dafny AST could be traversed.
While navigating through it, a symbol table is built in the form of a tree.
Each symbol is a tree node and stores it's child nodes during visitation.
Aside child-parent relationships, symbol usages are counted too, as well as declarations are resolved.
Thus, every occuring name segment in the Dafny code contains the following information:
\begin{itemize}
    \item Which symbol is my parent? For example, this could be a method body or a while loop, or just a block scope introduced by \code{\string{$\ddots$\string}}
    \item If I am a declartion, where am i used?
    \item If I am not a declaration, where am i declared?
    \item If I contain a body, which symbols are declared within my body?
    \item If I contain a body, which symbols occur at all within my body? This is, declarations and usages.
\end{itemize}
Thus, a feature like goto defintion can just call the information about the declaring symbol and the cursor can jump to it.
Thus, compared to the preexisting features, the following improvements could be acheived:
\begin{itemize}
    \item Goto Definition works now with respect to scopes and will not just jump to the first name match.
    \item Auto completion works with respect to scopes and will also guess whetever the user is interested in a class, for example after a \code{new}.
    \item Code Lens no longer trivially counts name matches.
    Instead it shows correct usage counts and previews can be displayed.
\end{itemize}



\subsection{Outlook}
While the quality of the features, as well as the general code quality could be massively improved, the functionality of the project could be improved even further.
Ideas include:
\begin{itemize}
    \item Automatic generation of contracts
    \item Debugging
    \item Create clients for other IDE's.
\end{itemize}
Aside the widening of the feature range, it is definitely necessary to complete the visitor, which currently only traverses the most important AST node.
This was due to the limited time frame of the bachelor thesis.
Nevertheless, the plugin is of a nice quality and may be deployed into the VSCode market place.
Thus, future students can work with it and make their first steps in the Dafny programming language using our plugin.








\intnote{2do: iwo den Satz "Zielgruppe die HSR Studenten" einbauen.
"Messbarkeit von Erfolg." habs ams chluss kurz aber nja, evtl noch etwas deutlich iwo.}
\intnote{2do: allfeatures in one bild? hat da snoch platz überhaupt xD}
