\section{Introduction}

\label{section:introduction}
This chapter describes the initial solution of the project, as well as the motivation and the goals of the thesis in more detail.
To provide the reader with the necessary context, the technologies touched by this bachelor thesis are explained at the beginning.
This mainly concerns Dafny and the language server protocol (LSP).

\subsection{Dafny}
\label{section:introduction:dafny}
Dafny is a compiled language optimized to prove formal correctness \cite{dafnyWiki}.
It bases on \textit{Boogie}, which uses the \textit{Z3} automated theorem prover for discharging proof obligations \cite{dafnyWiki}.
That means, a programmer can define a precondition - a fact that is just given at the start of the code.
The postcondition on the other hand is a statement that must be true after the code has been executed.
Just as the precondition, the postcondition are both defined by the programmer.
In other words, it can be proven, that under a given premise, the code will manipulate data only thus far, so that also the postcondition will be satisfied.
Dafny will formally ensure this.
If it is not guaranteed that the postcondition holds, an error is stated.\\

The following code snippet shows an example.
The value \code{a} is given, but it is required to be positive.
This is the precondition.
In the method body, the variable \code{b} is assigned the negative of \code{a}.
We ensure, that \code{b} must be negative, which is the postcondition.
\begin{lstlisting}[language=dafny, caption={Simple Dafny Example}, captionpos=b, label={lst:simpleDafnyExample}]
method Negate(a: int) returns (b: int)
requires a > 0
ensures b < 0
{
    b := -a;
}
\end{lstlisting}
This example is of course trivial.
In a real project, correctness is not always that obvious.
With Dafny, a programmer can be sure if the program is correct in a logical way.
Since the proof is done with formal, mathematical methods, the correctness is guaranteed.\\

If Dafny is unable to perform a proof, the user can assist by creating lemmas.
Lemmas are mathematical statements.
For example, a lemma could be that a factorial number is never zero.
If we define a simple function \code{Factorial}, and afterwards divide through the result of \code{Factorial}, Dafny will state that this might be a division by zero error.
But if we assert that a factorial number can never be zero, verification can be completed successfully.
\begin{lstlisting}[language=dafny, caption={Lemmas}, captionpos=b, label={lst:lemma}]

function Factorial(n: nat): nat
{
    if n == 0 then 1 else n * Factorial(n-1)
}

lemma FactorialIsPositive(n: nat)
ensures Factorial(n) != 0
{}

function Foo(n: nat): float
{
    FactorialIsPositive(n);
    100 / Factorial(n)
}
\end{lstlisting}


\subsection{Language Server Protocol}
The language server protocol (LSP) was created to unify communication between an integrated development environment (IDE) and a language server.
It specifies requests, such as auto completion, rename or go to definition.
If the user performs an action like rename, the IDE will send the proper request to the language server.
The message format is specified by the LSP and bases on JSON.

The language server is responsible to calculate a proper result.
For the example of a rename request, the answer contains the information where to apply the renaming.
It is the task of the server to analyze the source code and provide a rename response with respect to language specific rules.\\

Since the language server is independent of the client, a language server can be used from within multiple IDE's.
To provide support for another IDE, just the client has to be coded.
Since all logic is contained within the server, this can be done with minimal effort.
A developer only has to set up the connection to the language server and has to implement the display logic for the newly supported IDE.

The most important requests the LSP supports incorporate:
\begin{itemize}
    \item Transfer code to server
    \item Show errors and warnings
    \item Rename a symbol
    \item Go to the definition of a symbol
    \item Show auto completion suggestions
    \item Perform a refactoring
    \item Show usages of a code item
\end{itemize}

These are features that are commonly used by programmers, independent of the language.\\

Aside these standard features, Dafny has the option to show a counter example, if a postcondition is violated.
This request is not natively supported by the LSP.
However, own custom requests and data responses can easily be added to the LSP.
These have to be handled separately within the client though, since the LSP is not automatically displaying the result. \\

Because of these advantages our work is based on this protocol.
Exactly which features have to be implemented in the LSP is discussed in the analysis part of this thesis.

\subsection{Initial Solution and Motivation}
\label{section:introduction:initialsolution}
In a previous bachelor thesis by Markus Schaden and Rafael Krucker, a LSP client-server infrastructure for Visual Studio Code was created to support Dafny \cite{ba}.
The plugin was particularly appreciated by the "HSR Correctness Lab" \cite{correctnessLab} to make coding in Dafny easier.
Its development was continued within the HSR under the lead of Fabian Hauser.
In the following text, the existing project will be referred as the \textit{pre-existing project}.
The pre-existing project offered a LSP-client for Visual Studio Code, which connected to a language server.
Both, the language server and the plugin, were written in TypeScript.
To communicate with the Dafny backend, the language server used a proprietary JSON-interface.
Information provided by Dafny was parsed from the console.
Functionality was therefore limited to the Dafny console output.
Due to the preparation of this console output and subsequent preparation for LSP communication,
it was very complex and time consuming to implement additional features. \\

Marcel Hess and Thomas Kistler continued the development of the project within a semester project \cite{sa}.
The language server was migrated to \CsharpWithSpace and could be integrated into the Dafny backend
as you can see in figure \ref{fig:oldBAvsSA}.
Any Dafny functionality was directly accessible and the proprietary JSON-interface, as well as console parsing, could be omitted.
All features were reimplemented to satisfy the new architectural layout.
The result of this semester thesis will be referred as the \textit{prototype}.
If the actual text of the preceding semester thesis is targeted, it will be called the \textit{preceding semester project} throughout this text.

\begin{figure}[ht]
    \centering
    \includegraphics[width=\textwidth]{02_oldBAvsSA.png}
    \caption{Architecture before (left) and after (right) the prototype was created}
    \label{fig:oldBAvsSA}
\end{figure}


\subsection{Goals}
At HSR, Dafny is taught within the course \textit{Software Engineering}.
Students have to make their first steps in Dafny, using a simple tutorial.
To provide the students with the necessary support, they shall profit by common IDE features like
\begin{itemize}
    \item Error highlighting
    \item Compile and run
    \item Auto completion
\end{itemize}

On the other hand, the plugin shall also be used in a professional environment.
Thus, more advanced features have to be supported.
To facilitate the development of further features, the code shall be refactored to achieve a clean state.
This especially involves a clear architectural layout with dependencies that only point downwards.

Aside a clean architecture, a major goal is the creation of a custom symbol table.
This is necessary to make features like \textit{go to definition} or \textit{auto completion} possible at all.
It also allows the development of further features.
For example, if every the symbol table provides information about the scope-depth, the indentation width can be deduced for a feature like auto formatting.\\

The symbol table shall further contain a well organized tree structure, so that navigation can be done efficiently.
Each symbol shall have direct links to it's parent, children and declaration for very fast access.
Compared to the prototype, the pre-existing implementations can be extracted into the symbol table, reducing the lines of code (LOC) and increasing performance.


Ziele, Aufgabenstellung
Plattform Independence
Setup, easy installation
UsabilityTest Code
IDE Independence, Wiederverwendbarkeit
Feature Richness
Einfache Erweiterbarkeit
....
Ziele: project cleanup
Repository Splitting
Merge Back ready state (Rebase)
General Code cleanup
SonarQUbe run, fixes
clean loggin
features verbessern gemäs SA als AUsblicke festgelegt
symbol table implementieren, um gotodefinition zu verbessern sowie künftige Features

optional: neue features
hover information
codelens
refactorings like extract method
automated contract generation
