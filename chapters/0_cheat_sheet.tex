\section{Cheat Sheet (TMP)}
Dieses CheatSheet wird später wieder entfernt.

Hier ist Text auf einer neuen Zeile. \\

Jetzt ist Text mit einer Zeile Zwischending.

Wegen dem rechten Linebreak füllt diese Zeile den ganzen horizontalen Raum. Nach dem Linebreak \linebreak  kommt eine neue Zeile.
Neue Zeile. \bigskip
Bigskip scheint einfach eine neue Zeile zu sein.\\

Ich bin \textbf{fett} und \textit{kursiv}. Inline code geht mit \texttt{texttt} oder analog mit \code{code} weil man die 3t's eh versaut. \Csharp muss man escapen mit dem command \textbackslash Csharp.

\subsection{Untertitel}
\subsubsection{Das ist die tiefste Titelebene}
Ich bin Text.
\begin{figure}[H]
    \centering
    \includegraphics[width=100mm]{musterBild}
    \caption{My caption}
    \label{fig:bsp}
\end{figure}
Davor ein Bild. \\
Mehr dazu in Abbildung \ref{fig:bsp}.

\subsection{Quellen}
Und das wäre ein zweiter Absatz \cite{ba}.

Wie einer auf 20min sagte:\cite{dev}
\begin{quote}
Immer mehr europäische Länder verhängen im Kampf gegen das Virus eine Ausgangssperre.
\end{quote}

Beachten sie die Fussnote\footnote{Ich bin die Fussnote}

\subsection{Aufzählung}
\begin{itemize}
    \item Erstens
    \item Zweitens
\end{itemize}

\begin{enumerate}
    \item Erstens
    \item Zweitens
\end{enumerate}

\begin{description}
    \item Erstens
    \item Zweitens
\end{description}

\subsection{Tabelle}
\begin{figure}[!h]
    \centering
  \begin{tabular}{| c | c | c | c |}
    \hline
    Col1 & Col2 & Col2 & Col3 \\ [0.5ex]
    \hline\hline
    1 & 6 & 87837 & 787 \\
    \hline
    2 & 7 & 78 & 5415 \\
    \hline
    3 & 545 & 778 & 7507 \\
    \hline
    4 & 545 & 18744 & 7560 \\
    \hline
    5 & 88 & 788 & 6344 \\ [1ex]
    \hline
  \end{tabular}
  \label{tab:tabbsp}
  \caption{My table}
\end{figure}
Das war also Tablle \ref{tab:tabbsp}.

\subsection{Code}
Fogelnd Code in \Csharp
\lstset{style=sharpc}
\begin{lstlisting}[caption={My Caption}, captionpos=b, label={lst:codebsp}]
        public void SendInformation(string msg)
        {
            SendMessage("INFO", msg);
        }
\end{lstlisting}

\lstset{style=dafny}
\begin{lstlisting}[caption={My Caption}, captionpos=b, label={lst:codebsp2}]
        DafnyCode() {}
\end{lstlisting}
Da wär jetzt so code, siehe Listing \ref{lst:codebsp2}.

The user needs to write \texttt{this} in front o fthe variable \texttt{myVariable}.

\subsection{Referenced Section}
\label{section:my}
You can read more about references in section \ref{section:my}

