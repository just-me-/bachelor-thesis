\section{Conclusion}


\subsection{Project Summary}
Project ist success. Nice plugin created. Due to timely resasons, symbol table nciht ganz fertig, kann aber noch erledigt werden. Die Grundsteien sind gelegt.
Für erste steps in dafny, absolut sufficient.

Es hat einfach an der Zeit gefehlt, aber der Weg ist offen das projekt weiterzuentwickeln. man müsste halt nur die tabl enoch fertig machen, udn dann hat man schonmal ein geiles teil.
danach könnte man fast den ganzne lsp standard umsetzen, z.b. gibt es den highlight request, da kann man einfach symbol table frage: gib mir alle dinger beim cursor, highlighte die, fertig, aber mehr dazu im foglenden kapitel.

\subsection{Deployment}
Fabian hat das Deployment als preview verison vorbereitet, man kann das auch durchaus machen, wenn man das projekt im rahmen der SE Tutorials benutzt.
für ein breites publikum müsste man eig schon noch die symbol talbe fertig machen.
ist halt bA, time boxed, da muss der scope variable sein (Keller ziteieren)
kann man aber raushauen das plugin, ist gut genug.

\subsection{outlooks}
Symbol table fertig machen.


LSP bietet noch scheisse viel shit: (eigenes kapitel)
Goto Definition, Goto Declaration unterscheiden, dann gibht es noch goto implementation.
Find References könnte relativ einfach gemacht werden, textDocument/references. Dazu auch highlight request, wo einfach alle referenzen highlightet werden. DAs wären recht simple features. Wir wollten uns aber eher um die core shit kümmern drum haben wir eher qulität statt quanittät gemacht auch wenn wir diese feature ruckzuck gehabt hätten.
symbol table auch lesen alle files und so auch nicht imported und dann automatisch das machen gibt sogar requests dafür vom lsp "workspace/symbol" wo man filter mitgeben kann (wtF)
auto completion: triggeren nach dot und new.
aut ocmpletion: über args springen, das geht mit dem textDocument/signatureHelp Request - war unser ziel aber konnten uns zeitlich nicht merh darum kümmern
auto format wär so was... da wir ja immer den scope kennen jetzt mit der table, könnte man auch die depth  bestimmen, udn anhand von dem wievicel indentation sein muss. eigentlich simple.


Dann eigenes kapitel dafny specific:
auto contract generation (ev sogar AI thema, sind wir halt nicht azu gekommen)
extraft method könnte man anhand der symbol table machen - müsste kucken welche symbole sind im scope,w elche muss ich als argumente haben, welche nicht, mimi, dann einfach unten methode hinklatschen. und den code reinkopeiren. durchaus machbar. wir hatten halt einfach keine zeit.
debugging sehr kompliziert, eigene BA, nach wie vor wäre natürlich fancy shit.

Server Client ist jetzt ja noch besesr getrennt,d .h. integartion in andere IDE wie exlipse ist nun noch simpler. speziell ist halt alles was non-lsp-standard ist, also compile und counter example, insbesondere da mit custom args. das müsste man halt noch gesondert behandeln.






\subsection{Achieved Improvements for Further Development}
\intnote{so motto: goals of this ba: auflisten von dem zettel da und alles abhaken. voll, auch so, git commit messages wurden angekreidet, jetzt besser geamcht (ein bisschen).}
Fabians Feedback aus der SA... neues Review. "Tu Gutes und sprich davon". -> vlt auch eher bei results "es wrude gsagt, das muss gemacht werden -> haben wir gemacht"
Aktuell bin ichs ogar dafür, das nach conculsuison zu moven. können sicher auch fragen am meeting.





\subsection{Metriken WO KOMMT DAS KAPITEL HIN???}
Performance Verbesserungen, Anzahl Code Lines, reduktion der lines/code anzahl / methodenanzahl.
"Wie viel chelaner" unsere lösung geworden ist. wie viel schneller. sowas ist wichtig fpr t.corbat. 
eventuell auch conclusion? ich weiss es wirklich nicht :O doch nach results? implementation?

.