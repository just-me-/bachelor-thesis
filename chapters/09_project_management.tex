\section{Project Management}
\label{section:project_management}


metriken aus dem alten und dem neuen sonar cloud?? wäre easy und fancy.  bei results.


projetplne; git. er mehr server/symbotabte, ich mehr lcint.
basic LSp beide. /featuresd provider und co. essenzielles im pair programming. 

todo

QualitätsManagement: Static code analyze: SonarLint. \cite{sonar-lint}
Wir haben zwar SonarQube. Aber analysiert erst im CI.
damit kann man auch lokal eine statische codeanalyse laufen lassen sprich fehler werden sofort lokal beid er
entwicklung bereits angezeigt. einfacher zu beheben.


Die andern:
Proejtk Plan, Milestones, Risikomanagement, Abweichungen vom Proejkt Plattformen
Projekt Homepage
Zeitrapport
Code MEtriken



sonar, prettier, sonarlint,

kapitel von der sa und alten ba durchscrollen.
diagramm aus präsentation fuer google.

Continuous Integration (CI)
Projekt management: darum haben wirs getan, qualität
According to the project plan, we wanted to resolve all CI-issues at the beginning of the bachelor thesis, so that we can then profit by a supportive workflow.



\subsection{Meetings}

\subsection{Time Management}
\subsubsection{Project Management}

%\subsection{Scope - results? bzw eig jedes kapitel? - weg}
%\subsection{Metrics and Code Quality Aspects - result gem corbat}
%also heir so: wir haben sonar genutzt, für qualität. aber der report bei result.

\subsubsection{Code Reviews}

\subsubsection{Backend}
Lines of Code
Other Metrics
SonarQube (new)

\subsubsection{Frontend}
SonarQube

\subsubsection{Test Coverage}

\subsubsection{Commit Activities}


\subsection{Infrastructure}
