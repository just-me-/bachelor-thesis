\section{Result}
In this chapter we describe the achieved results of our work.
On the one hand, this concerns the features offered by the plugin
(and accordingly by the implemented Language Server),
but on the other hand also the architectural improvements to achieve further development of the project for other developers.

\subsection{Features for the Plugin User}

\subsubsection{Compile}
\subsubsection{Counter Example}
\subsubsection{Code Verification}
\subsubsection{CodeLens}
\subsubsection{Automatic Completion}
\subsubsection{Hover Information}
\subsubsection{Rename}

\subsection{Achieved Improvements for Further Development}
Fabians Feedback aus der SA... neues Review. "Tu Gutes und sprich davon". -> vlt auch eher bei results "es wrude gsagt, das muss gemacht werden -> haben wir gemacht"

\subsubsection{Metriken}
Performance Verbesserungen, Anzahl Code Lines, reduktion der lines/code anzahl / methodenanzahl.
"Wie viel chelaner" unsere lösung geworden ist. wie viel schneller. sowas ist wichtig fpr t.corbat. 

\subsection{outlooks}
Nicht erreichtes, was nicht bei den Features etc selbst schon genannt wurde.

\subsubsection{Mono Support for macOS and Linux}
Eines der Kernzeiele war es, Support fuer mehrere Plattformen zu bieten. Dh nebst Windows auch macOS und Linux.
Da wir in unserer SA von Core auf Framework umsteigen musste, stand fest, dass wir mono fuer den Support auf Linux und macOS brauchen.
(warum in der SA; plficht wegen dafny core. was ist mono)

Leider funktionierts nicht.
Anssaetze die wir probiert haben. verschiedene mono versionen, angefragt im slack. antwort erhalten?
github issues: allgemein probleme mit lunux/mac weil primaer auf windows und gar nicht auf mac getestet wird. (heikle aussage selbs tmit quelle)

\cite{sa}
\cite{mono-slack}
\cite{mono-git}
