\section{Results}
\label{section:results}

 \intnote{Corbat: Da fehlt iwie ein kapitel für Comparison loc for implementation of features with symbol table present? loc symbol table? 2000. Run -> Code analysis metrics iwas for project.}

As described in our objectives, our work has three basic stakeholders. \\

On the one hand there are the students of the HSR, who want to get familiar with Dafny.
For these students, a usefull plugin should be offered to make the development with Dafny easier. \\

On the other hand, there are the developers who want to extend the language server in the future.
This category also includes developers who want to integrate other IDE's by additional clients.\\

Last but not least, anyone willing to program with Dafny may download the plugin from the VSCode marketplace.
Since we wanted to make a contribution to the Dafny community with our work,
it was always a big concern for us to achieve a quality level, so that it can be published.\\

This chapter describes the results achieved and further prospects for each target group.

\subsection{Features for Dafny Developers}
In this section, the achieved feature set of the project is described and critically reflected.
The current version of the plugin supports the following functionality:
\begin{enumerate}
    \item Syntax Highlighting
    \item Code Verification
    \item Compilation
    \item CounterExample
    \item Hover Information
    \item GoToDefinition
    \item Rename
    \item CodeLens
    \item AutoCompletion
\end{enumerate}

\subsubsection{Syntax Highlighting}
\label{section:result:syntaxhighgliht}
Syntax highlighting is realized by a given Dafny grammar file.
The file contains regex expressions defining the highlights.
It is provided by Dafny \cite{syntax_update} and could simply be downloaded.
The feature was already implemented in the prototype, thus no further actions aside updating the grammar file had to be done.
The following screenshot shows how syntax highlighting looks inside Visual Studio Code.

\begin{figure}[H]
    \centering
    \includegraphics[width=10cm]{results/syntax_highlighting.png}
    \caption{Syntax Highlighting}
    \label{fig:result_syntax_highlight}
\end{figure}

As you can see, keywords like \textit{method}, \textit{returns}, \textit{requires} and \textit{ensures} are marked in purple.
Types like int are printed in blue and comments become green.
Symbols, such as classes and methods, are displayed in a brownish color.
Just these simple rules increase the readability significantly.

\subsubsection{Verification}
Verification was already implemented in the prototype, too.
At the start of the project, the feature held some major flaws.
\begin{itemize}
    \item It only reported logical errors.
    \item Syntax errors were not announced.
    \item Warnings were ignored.
\end{itemize}

The code just invoked the \code{DafnyTranslationUnit}, taken over by the pre-existing project.\\

Verification was reworked completely during this project.
First of all, Dafny code itself was analyzed to get a better understanding of how Dafny compiles its code and how errors are reported.
Those errors could finally directly be extracted out of the Dafny error reporting engine.
Thus, the user is now informed not only about logical errors, but also about syntax errors.
Furthermore, warnings and information diagnostics are now also reported.
Reporting warnings is something the pre-existing project did not do and was actually already issued on the official Dafny git repository \cite{dafny_noWarnings}.
With the symbol table, it is now also possible to underline complete code blocks,
instead of just single symbols which were barely visible at the beginning of the project as you can see in figure \ref{fig:error_whole_block}.
This shows well how the quality of the language server could be improved.

\begin{figure}[H]
    \centering
    \includegraphics[width=9cm]{results/error_whole_block.png}
    \caption{Underlinding Before (Top) and After (Bottom)}
    \label{fig:error_whole_block}
\end{figure}

The feature directly invokes Dafny's compile engine, thus works quite solid and scales automatically with future Dafny features.

\begin{figure}[H]
    \centering
    \includegraphics[width=15cm]{results/warningAndInformation.png}
    \caption{Reporting of Warnings and Information Objects}
    \label{fig:result_warnd_info}
\end{figure}

\subsubsection{Compile}
The compile feature is strongly connected to the verification process.
Prior to compilation, the whole Dafny project has to be verified anyway.
Thus, since verification yields a precompiled \code{DafnyProgram}, the buffered result can be used to invoke the Dafny compiler.
This makes the compilation process very snappy and responsive.\\

If the code contains errors, the verification process already failed and compilation can instantly be denied.
However, if the code is fine, the precompiled \code{DafnyProgram} just has to be translated, which can be done quickly.

The user also has the option to apply custom compilation arguments.
These can be directly set within VSCode.
\begin{figure}[H]
    \centering
    \includegraphics[width=15cm]{results/compile_customarg.png}
    \caption{Custom Compilation Arguments}
    \label{fig:compilation_custom_args}
\end{figure}

Custom arguments are directly handed to the Dafny options parser and are applied within the Dafny engine.
Since compilation uses the precompiled Dafny program, compilation arguments affecting the verification process have no effect.
This is something that could be resolved by just restarting the verification process if custom arguments were given.\\

In the prototype, compilation was implemented by just starting a sub-process, launching \code{Dafny.exe} with custom arguments given.
To obtain compilation results, the console output of the sub-process was parsed and reported to the user.
That solution was obviously not integrated at all.
The current implementation improved that significantly, is now completely integrated, both, in terms of invocation and result reporting.

\begin{figure}[H]
    \centering
    \includegraphics[width=10cm]{results/compile_contextMenu_feedback.png}
    \caption{Compilation Context Menu and User Feedback}
    \label{fig:compilation_stuff}
\end{figure}


If the user chooses the option to \code{Compile And Run}, the actual launch of the executable takes quite a long time - up to about 20 seconds.
However, this delay is not related to the language server or the compilation process.
If the executable is launched manually within another console outside of Visual Studio Code, it also takes long until the program starts.
The effect just occurs the first time an executable is launched.
The reason for that could not be investigated.

\subsubsection{CounterExample}
Providing CounterExamples was already possible in the prototype.
A major flaw was, that the representation of the CounterExample was quite complex and not intuitively readable.
Thus, it was a goal to ease the CounterExample representation.\\

For this, the related \code{model.bvd} file was studied.
It is quite a  cryptic file and getting a comprehensive understanding of it would be very complex.
However, an \code{inital state} was located which seemed exactly that part of information, that the user is interested in.
Thus, unlike previous in versions, only that \code{initial state} is considered in CounterExamples.
Furthermore, any unreadable representations such as \code{**myVar} or \code{TU!Val23} were omitted.
To allow the user to catch the information at first sight, obsolete brackets are also removed, and the minus sign is directly moved to the number.
The expression \code{((- 23))} is therefore reformatted into \code{-23}, making the term much more perceivable.

\begin{figure}[H]
    \centering
    \includegraphics[width=12cm]{results/ce_beforeafter.png}
    \caption{CounterExample Representation Before (Top) and After (Bottom)}
    \label{fig:ce_beforeafter}
\end{figure}

Figure \ref{fig:ce_beforeafter} shows a comparison of the CounterExample feature between the initial state and at the end of the project.
The representation is much cleaner and easier to immerse.
Also note, that more room is given so that is does not have a clumsy effect on the user's eye. \\

This feature was also improved on the client side.
The user has the option to configure the color scheme of the CounterExample representation, if he does not like the default colors.
The default colors are chosen with respect to the user's base color theme of VSCode (dark or light mode).
As in figure \ref{fig:ce_beforeafter} the light theme is shown,
the following figure \ref{fig:ce_dark} shows the dark mode colors.

\begin{figure}[H]
    \centering
    \includegraphics[width=8cm]{results/ce_dark.png}
    \caption{CounterExample Dark Theme}
    \label{fig:ce_dark}
\end{figure}

The CounterExample will correctly adjust if the user continues to work on the code and vanish, once the problem is resolved.
If the user switches between windows, the CounterExample state is buffered, so that once the user switches back to the original window, the CounterExample will be shown again.

\subsubsection{Hover Information}
Hover Information displays a set of information, whenever the user hovers with the mouse cursor over a code symbol.
The feature itself was very simple to implement, since it receives all necessary information from the symbol table.
It was additionally implemented as an exemplary feature, to show off how easy some LSP functionality can now be implemented using the new symbol table.
Similar to hover information, other features like text highlighting can be added as well in the future, which is discussed in chapter \ref{section:conclusion}.
Hover may not provide much useful information to the user, but it is still a nice gimmick.
For example, in figure \ref{fig:hover} the user can actually find out, where his \code{field} is declared if he is unsure.
\intnote{bild noch updaten}
\begin{figure}[H]
    \centering
    \includegraphics[width=\textwidth/2]{results/hover.png}
    \caption{Hover Information Example todo das bild ist altes hover}
    \label{fig:hover}
\end{figure}

\subsubsection{GoToDefinition}
Compared to the prototype, GoToDefinition has been significantly improved.
Prior to this bachelor thesis, the feature just scanned all code for the first name-match and reported it as the definition.
This did neither work if multiple symbols with the same name occurred, nor if the declaration was placed after the first usage in the code.
Both cases are now handled well by the symbol table engine.\\

Runtime could also be improved, since the symbol table generator scans for possible declarations scope by scope, and doesn't just iterate over all symbols.
Last but not least, in the prototype, the cursor had to be at the beginning of a symbol to recognize the symbol at all.
This major flaw has of course been overcome.
The cursor can now be at any character within a symbol.\\

Technically, the feature is misnamed.
A better name would actually be GoToDeclaration, since it jumps to the symbol declaration, not definition.
LSP would even offer a dedicated handler for this, \code{textDocument/declaration} \cite{lspspec}.
However, we kept the name to maintain the consistency of our work.
But perhaps a rename would be appropriate.

To increase the user experience with this feature, notifications are sent to the user
as shown in figure \ref{fig:goto}
if an error occurred or the requested symbol was already a declaration.
\begin{figure}[H]
    \centering
    \includegraphics{results/goto.png}
    \caption{GoToDefinition Error Reporting todo das bild ist falsch?}
    \label{fig:goto}
\end{figure}

Jumps can also be done across included files.
If the targeted file is not opened, VSCode will just open it inside the workspace.
It is also possible to go to the definition of \code{this}-Expressions, which will just jump to the class definition.
The feature works also across code that is not within a block statement, for example for expressions occurring inside an \code{ensures}-clause.

\subsubsection{Rename}
Rename was newly added to the language server.
It is a feature that could only be realized, since the symbol table provides all necessary information.
If the user wants to execute a rename, all dependent symbols are deduced correctly as you see in figure \ref{fig:rename}.

\begin{figure}[H]
    \centering
    \includegraphics[width=\textwidth]{results/rename.png}
    \caption{Rename of a Symbol}
    \label{fig:rename}
\end{figure}

It was embraced to provide as much user experience as possible for this feature, even if it is relatively simple.
In particular, this means if there is no renamable symbol at the cursor, the user is informed by a notification.
The algorithm will also check, if the new symbol name is valid.
This means, it must not start with an underscore or number, and it must not be any reserved Dafny word.
Also, we limited the allowed new name to alphanumerical characters, although wild unicode names would be allowed by Dafny.\\

This could even be driven further.
For example, it could be checked, whether a symbol with the same scope already exists.
Functionality for this is already available by the symbol table, since something similar is needed by autocompletion.
That is also something that could easily be added.

Since Dafny reserved words may change in the future, the wordlist is configurable for the end-user by adjusting a config file.
The process targets more advanced users and is described in the readme file.\\

Because the symbol table reports occurrences across imported files as well, the rename feature works well for included files.
However, if a file gets included elsewhere, it does not know about it.
This is, since no global symbol table is currently used for the entire workspace.
This issue and possible solutions were described in section \ref{section:globalSymbolTable}.\\

\subsubsection{CodeLens}
CodeLens is an excellent way for the developer to see which classes and methods were used how often - and where.
In order to be able to assign the references reliably and
also to consider the references across included files, the symbol table was necessary. \\

As shown in figure \ref{fig:result_codelens_references},
not only the references to the symbols are counted,
but also dead code is lifted by a special text.

\begin{figure}[H]
    \centering
    \includegraphics[width=\textwidth]{tmpPlaceHolderTodo}
    \caption{Codelens Shows the Number of Uses}
    \label{fig:result_codelens_references}
\end{figure}

If the user clicks on one of the reference texts, a popup window opens,
as figure \ref{fig:result_codelens_references_popup} shows.
On the right hand side of the popup window all references found are listed.
If the user clicks on the corresponding references, the corresponding code
location in the right file opens on the left side - at the location within the popup.
This is very convenient for developers,
because they no longer have to jump back and forth between files and you do not have to scroll unnecessarily.

\begin{figure}[H]
    \centering
    \includegraphics[width=\textwidth]{tmpPlaceHolderTodo}
    \caption{Codelens Shows References in Pop-up}
    \label{fig:result_codelens_references_popup}
\end{figure}

Just like Rename, the feature does not work for code that includes the current file
due to the missing global symbol table as described in section \ref{section:globalSymbolTable}.

\subsubsection{Automatic Code Completion}
Auto completion lists the available symbols in the current context on request.
Compared to our previous thesis, we could significantly improve this function.
Instead of displaying all found symbols in the opened file as before,
we can now use the new symbol table to consider the current scope for suggestions. \\

Our auto completion supports three basic types of proposals:
\begin{itemize}
    \item Completion after a dot was written.
    \item Completion after \code{new} was written.
    \item General suggestions.
\end{itemize}

If a dot was written, only declarations of the class in front of the dot are suggested.
If a new was written, only available classes are listed.
When none of the above is the case, all available symbols in the current scope will be displayed.
The three cases are shown in figuer \ref{fig:result_completion}

\begin{figure}[H]
    \centering
    \includegraphics[width=\textwidth]{results/completion_allinone.png}
    \caption{AutoCompletion}
    \label{fig:result_completion}
\end{figure}

\textbf{Additional Usabillity Sugar}\\
As a little extra to increase the user experience, when inserting a suggested function or method after a dot
or a class after a \code{new},
opening and closing brackets are automatically added.

However, the parameter list of methods is not yet proposed.
Technically it would be possible, as described in chapter
\ref{section:implementation:core:completion} \textendash{} \nameref{section:implementation:core:completion}.
This would further enhance the user-friendliness as proven in the usability test.

\subsection{Simplicity of Further Development}
This section describes architectural improvements,
which increase maintainability and facilitate future extensions of the project.

\subsubsection{Symbol Table}
Within this project, a new symbol table was implemented.
The prototype was already working with a primitive approach of a symbol table, that worked only with strings.
This is no comparison to the new symbol table, which works with proper objects.
Related information, such as the parent symbol, is now directly accessible.
Any string parsing could be completely omitted compared to the prototype.
This leads to better performance and, above all, to reliable symbol references.\\

All of the previous features, with the exception of AutoCompletion, do not contain much logic themselves.
They just request information from the symbol table and report it back to the client.
The symbol table allows the implemention of new features, such as CodeHighlight and even AutoFormat with ease.
This is a major gain for the future of the language server.\\


While the features are implemented quite robust, they only work as long as the symbol table provides proper information.
Whenever the symbol table fails, the underlying features will also produce nonsense.
Thus, a correct symbol table is crucial for correct functionality.\\

Dafny is a programming language offering a lot of features.
Aside common object oriented features, also functional programmatic features are present.
Figure \ref{fig:analysis_dafnyASTStuff} shows all classes occurring inside the Dafny AST.
The writers are well aware, that the text in this figure is too small to be read.
The figure should indicate the humungous amount of AST-elements present in Dafny.
Many of them are not just inheriting from \code{Expression} or \code{Statement}, but from individual base classes.
Thus, implementing the visitor for all of them turned out to be massively time consuming.
Since the bachelor thesis is bounded by a limited time frame, it was necessary to limit the amount of Dafny language features we support.
We decided to lay our focus on the symbols needed by the Dafny tutorial, which is done within the software engineering at HSR.
This way, the basic concepts of Dafny are supported.
We could support about 50\% of the available AST-nodes.
However, as soon as the user starts to use more advanced language features, a symbol may be introduced within a scope that is not visited by the visitor.
If the user is using that symbol later on inside a common method body, the symbol table generator will be unable to locate the symbol's declaration and thus fail.
An overview of the supported expressions and statements was given in section \ref{section:analysis_dafnyASTStuff}.\\

It is subject of further development to complete the visitor.
Primarily, it lacks support for generics and custom datatypes.
Finishing the visitor should have very high priority in the future development of this project,
to provide the best possible user experience.

\begin{figure}[H]
    \centering
    \includegraphics[width=\textwidth]{results/symboltable_dafnyast.png}
    \caption{All Dafny AST Classes}
    \label{fig:dafnyASTOverview}
\end{figure}

\subsubsection{Server Architectural Improvements}
At the beginning of the thesis, the server's dependency graph was quite a mess.
During development, the code was constantly cleaned up and dependencies were resolved.

While the project is quite large and has a lot of dependencies, they could still be well organized.
As seen in figure \ref{fig:dependency_graph_2}, all dependency are now pointing strictly downwards only, just as it should be.
Within the picture, dependencies to the \code{Utilities} layer are excluded for more overview.
\begin{figure}[H]
    \centering
    \includegraphics[width=\textwidth]{dependency_diagram_server_final.png}
    \caption{Server Package Dependencies, generated by Visual Studio}
    \label{fig:dependency_graph_2}
\end{figure}

Aside the broad architectural layout, many small refactorings were made to keep the code clean.
This includes the creation of smaller classes and smaller methods with single responsibilities.
A good example for this is the config initialization, which was one single large class at first.
After refactoring, it was split into small, well organized sub-classes.\\

Any important classes, such as handlers, providers or classes related to the symbol table are programmed against interfaces.
This allows for optimal testing, for example by implementing the same interface with a fake instance.

\subsubsection{Client Architectural Improvements}
The client was completely restructured.
Isolated, small, preferably encapsulated components were programmed.
Components are now no longer directly dependent on each other but program against interfaces.
Modules were created and these export only the most necessary interfaces and components to the outside.
If complex logic would have to be implemented for the client at a later date due to new features,
this encapsulation by interfaces is ideal for writing tests. \\

This resulted in less encapsulation and higher coherence.
These two motivations also encouraged the complete isolation of VSCode modules.
We isolated the VSCode components in a separate module if possible.
This minimizes the effort to migrate the plugin to other IDEs.


\subsection{Publication of the Plugin}
The final plugin has been published to the Visual Studio Code Marketplace as a preview version \cite{our-dafny-plugin}.
The installation is very easy and runs automatically.
Any Windows user can install it \textendash{} try it out and leave feedback.\\

This chapter describes further aspects related to the release.

\subsubsection{Pull Request to Dafny}
Our Dafny language server has reached a point where it technically can be merged back into the original Dafny project \cite{dafny_lang_github}.
Unfortunately, they released a new version shortly before the end of the thesis.
Thus, a rebase is necessary first.
Since Dafny changed the underlying .NET Framework version, the rebase was executed as a proof-of-concept, but was not carried out to avoid bugs.\\

There is also still a problem with platform independence as described in in chapter
\ref{section:implementation:mono} \textendash{} \nameref{section:implementation:mono}.
Although this was a major goal, it could not be achieved.
Before the pull request, those issues may be resolved.


\subsubsection{Publication to the VSCode Marketplace}
We merged our GitLab repository to a dedicated GitHub repository.
From there, it is uploaded to the VSCode marketplace.
This process uses the GitHub pipeline \code{release-marketplace}, which
was adapted by Fabian Hauser \cite{our-dafny-plugin-github-publish}.

Once the pull request to the Dafny project is complete,
the preview released plugin can completely replace the original plugin.

\subsubsection{Download of the Dafny Language Server}
In the pre-existing project, Dafny was downloaded from GitHub \cite{dafny_lang_builds}.
As discussed in section \ref{section:implementation:client:download} \textendash{} \nameref{section:implementation:client:download},
we have written a custom interface for this original process.\\

Thanks to the interface, we can currently download the Dafny language server from a private server \cite{client-serverStringResources}.
Once the pull request to Dafny is accepted, the download process can then be easily adjusted to download everything from Dafny directly.\\

We decided on this simple variant in order to keep the effort for this temporary solution as low as possible.
Alternatively, this process could have been automated with the GitLab CI using the GitLab API \cite{gitlab-api} to download the CI artifacts.
The artifacts could also be made available through the CI process as a static website with GitLab Pages \cite{gitlab-pages}.

\subsubsection{Easy Installation}
Mit screens vom Installieren.
todo

\subsection{Performance of the Language Server}
To measure performance, a little algorithm was written that creates a pseudorandom Dafny file.
Within a method, code lines are generated, either containing
\begin{itemize}
    \item A variable definition: \code{var v142 := v16;}
    \item A variable access: \code{print v142;}
    \item A block scope: \code{while (true) \{ \dots}
    \item Ending a blockscope: \code{\}}
\end{itemize}
The chance to create a variable or to create print statement is 90\%, thus the generated files will contain about 90\% variable name segments, that have to be resolved.
This challenges the symbol table quite a bit.
A tree is still grown due to the while-blocks, so not all symbols are within the same tree branch.
Since the \code{textDocuemt/didOpen}, which triggers the symbol table instanciation, is not awaitable, a random LSP request was sent after the opening to wait until all actions are finished.
The test has shown the following results:

\begin{itemize}
    \item 100 LOC: 500ms
    \item 1000 LOC: 1200ms
    \item 10'000 LOC: 18'000ms
\end{itemize}

 \intnote{grafik in excel erstellen, screenshot, fix rein. todo }

This scheme follows a $O(n*log(n))$ scheme, which was expected.
While 18seconds seem quite high, consider that 10'000 lines of code is huge.
When working on the \code{DafnyAST.cs} file, which also contains 10'000 LOC, even Visual Studio struggled.
Practical tests have shown that everyday usage of the plugin is pleasant.
It generally reacts fast enough.
All methods are \code{async}, so that the user is never blocked in his activity.

\subsection{Metrics and Quality Measures}
\label{section:results:metrics}
This chapter describes the most important metrics and quality measures for the server and the client.
The results are also compared to the prototype.

\subsubsection{Server}
The language server consists of 4'000 lines of code. The following report is generated by Visual Studio.
\begin{figure}[H]
    \centering
    \includegraphics[width=\textwidth]{results/server_metric.png}
    \caption{Server Package Dependencies, generated by Visual Studio}
    \label{fig:dependency_graph}
\end{figure}
Note that \code{Resources} contains generated files and should not be counted.
Sonar will report 4'000 LOC.

Compared to the prototype, this is an increase of about 200\%.
Aside the increased functionality, this is mainly to explain by the extensive and repetitive code necessary for the visitor within the \code{SymbolTable} namespace.
Just to create a single symbol, aobut 20 LOC are necessary as seen in chapter \ref{chapter:implementation}.\\

Interesting is the comparison of the code base required for a feature like GoToDefinition.
The new  \code{DefinitionsProvider} just uses 38 lines of code.
However, only about 5 are used for the core logic, all others are just error handling and user messaging.
This is a major improvement comapared to the pre-existing project.

The Sonar report for the server only detects about 15 issues.
The security aspects were ensured to be no problem, since the server is running on a local machine.
We did not want to remove the remaining smells.
For example, Sonar would like us to put all \code{Visit} and all \code{Leave} methods next to each other.
However, we prefer to have the \code{Visit} and \code{Leave} methods next to each other, when they belong to the same AST node.

To sonar report is shown in figure \ref{fig:sonarserver}

\begin{figure}[H]
    \centering
    \includegraphics[width=\textwidth]{results/server_sonar.png}
    \caption{Sonar Server Report}
    \label{fig:sonarserver}
\end{figure}


\subsubsection{Client}
For the Client we have direct comparisons between the old plugin and the new plugin thanks to SonarCloud.

wie begründen mit interfaces und strings die alle ausgelagert wurden... naja.
todo

uund wir haben doch einen neuen styleguide mit neuen umbrüchen oder?
