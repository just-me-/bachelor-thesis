\section{Management Summary}
\label{section:management_summary}

This chapter contains a brief overview of the bachelor thesis at hand.
First, the technologies Dafny and language server protocol are explained to provide the necessary context.
This is followed by a presentation of the objectives, results achieved and future prospects.

\subsection{Dafny}
\label{section:management_summary:dafny}
Dafny is formal programming language.
Within Dafny code, a developer has the option to state a so-called precondition.
This is a fact that must be true at the beginning of the code.
Once the code has been executed, the developer may state a postcondition.
The postcondition is, contrary to the precondition, something that is true at the end of code.
Dafny contains an internal engine to automatically prove postconditions.
This means, given the precondition holds, the code will manipulate data only thus far, so that also the postcondition is valid.
Dafny offers optimizations for such proofs as an advantage over other programming languages.

\subsection{Language Server Protocol}
\label{section:management_summary:lsp}
The language server protocol (LSP) is a specification for communication. \glsadd{lsp}
It regulates data exchange between an IDE and a so-called language server.
This ensures a separation between the integrated development environment (IDE) and the actual programming language support of the language server. \glsadd{ide}
With this standardized protocol, a language server can be accessed by different IDEs. \\

A plugin for the IDE "Visual Studio Code" (VSCode) was implemented during this bachelor thesis. \glsadd{vs_code}
Once a user writes Dafny code using the plugin, the code is then transferred to the language server.
The language server is now responsible for analyzing the code and provide proper replies to the editor whenever the editor requests something.
A request could, for example, be to show information about a variable.
The server has to provide all the necessary information to the IDE.\\

As a result of using the LSP, it is possible that Dafny can be supported by other IDE's with minimal effort.

\subsection{Initial Solution}
\label{section:management_summary:initialsolution}
In a previous bachelor thesis by Markus Schaden and Rafael Krucker, a LSP client-server infrastructure for Visual Studio Code was created to support Dafny \cite{ba}.
The plugin was particularly appreciated by the "HSR Correctness Lab" \cite{correctnessLab} to make coding in Dafny easier.
The language server and the Dafny backend were separated into two different components.
This had the disadvantage that they had to communicate over a proprietary interface.
This solution was not optimal, since the Dafny backend was not accessed directly.
Implementing new features was very time consuming. \\

In the preceding semester project \cite{sa}, the language server was merged into the Dafny backend, so that the proprietary interface became obsolete as you can see in figure \ref{fig:oldBAvsSA}.
Dafny could be accessed directly by the new software architecture.
Within this bachelor thesis, the work of the existing solution will be continued.

\begin{figure}[H]
    \centering
    \includegraphics[width=\textwidth/2]{02_oldBAvsSA.png}
    \caption{Architecture Before (Left) and After (Right) the Prototype was Created}
    \label{fig:oldBAvsSA}
\end{figure}

\subsection{Goals}
At HSR, Dafny is taught within the course \textit{Software Engineering}.
Students have to make their first steps in Dafny, using a simple tutorial.
To provide the students with the necessary support, they shall profit by common IDE features like:
\begin{itemize}
    \item \textit{SyntaxHighlighting} to make the readability more pleasant.
    \item \textit{ErrorHighlighting} to correct faulty code efficiently.
    \item \textit{CompileAndRun} to execute Dafny programs and check the output.
    \item \textit{AutoCompletion} to code efficiently.
    \item \textit{GoToDefinition} to be able to view definitions quickly.
    \item \textit{CounterExample} to efficiently remedy faulty pre- and postconditions.
\end{itemize}

On the other hand, the plugin shall also be used in a professional environment in the future.
Thus, more advanced features have to be supported.
To facilitate the future development of such features, the codebase shall be refactored to achieve a clean and maintainable state.
This especially involves a clear-cut architectural layout with well-organized dependencies.\\

Aside from a clean architecture, a major goal is the creation of a custom symbol table.
A symbol can be a variable, a function or a class.
The symbol table contains information about these items appearing in the Dafny code.
It shall allow very simple navigation within the Dafny code, for example to locate the declaration of a variable.
Once the symbol table has been generated, navigation within the code will be very easy for the language server.
This has the advantage that many features can be realized with almost no logic.
For example, the feature \textit{\textit{GoToDefinition}} just has to ask the symbol table where the definition of the symbol is.

Consequently, a well implemented symbol table will also facilitate the development of further features.

\subsection{Results}
%%intro
Within this bachelor thesis, the development of the pre-existing language server and its VSCode client was continued.
Significant improvements could be achieved, which are described in this section.

%%Features
The following features are supported as planned:
\begin{itemize}
    \item Syntax highlighting.
    \item Verification: highlighting of errors and warnings.
    \item Compilation of Dafny code and execute it.
    \item Providing counter examples.
\end{itemize}

Additionally, the following functions could be implemented thanks to the new symbol table:
\begin{itemize}
    \item \textit{AutoCompletion} suggestion.
    \item \textit{GoToDefinition}.
    \item \textit{CodeLens} counts the usages of classes and functions and lets the developer show the usages with one click.
    \item \textit{Rename} allows to rename symbols efficiently and reliably.
    \item \textit{HoverInformation} provides the developer with useful information about symbols.
\end{itemize}

\begin{figure}[H]
    \centering
    \includegraphics[width=\textwidth]{show_autocompletion_and_hover}
    \caption{The Features \textit{HoverInformation} and \textit{AutoCompletion} Profit From the Symbol Table}
    \label{fig:show_autocompletion_and_hover}
\end{figure}

In figure \ref{fig:show_autocompletion_and_hover}, a short code example is shown.
The code fragment would build a symbol table as it is abstractly represented in figure \ref{fig:symbol_table_for_autocompletion_and_hover}.
The features shown in figure \ref{fig:show_autocompletion_and_hover} such as \textit{AutoCompletion} and \textit{HoverInformation} directly use the offered symbol table.

\begin{figure}[H]
    \centering
    \includegraphics[width=\textwidth/2]{symbol_table_for_autocompletion_and_hover}
    \caption{Symbol Table to Represent all Symbols Inside the User Code}
    \label{fig:symbol_table_for_autocompletion_and_hover}
\end{figure}

Dafny developers can now benefit by a plugin, which provides a great user experience,
but also gained much robustness compared to the prototype from our previous project.\\

%%Dev Benefits.
Alongside the improvements in features, many internal aspects were also improved.
This benefits developers who want to extend the Dafny language server in the future.\\

A component called "Dafny translation unit" was completely re-visited and simplified.
The component accesses any Dafny functionality.
For example, instead of passing Dafny options as an array of strings, they are now set by
directly accessing Dafny's config class.
Any results provided by the Dafny translation unit are buffered for later re-use at compilation or to create the symbol table.
This makes the implementation significantly more performant.\\

%%ST
The targeted symbol table could be implemented for the most important Dafny language features.
Various challenges had to be accommodated, including the handling of default scopes, default classes, inheritance, external file import or variable hiding.\\

The symbol table opens the option to implement many more features than currently provided.
For example, the LSP offers a highlight request, marking occurrences of a symbol.
Any information required for this feature are already provided by the symbol table.
Thus, adding this feature would be very simple.\\

%%client
Besides changes in the server, the VSCode client is now as lightweight as possible.
This makes the adaption to other IDE's very simple.
The server is now able to create a symbol table containing any information required for the langue analysis features.
Pre-existing features and algorithms were improved to gain more reliability and a better user experience.

\pagebreak

\subsection{Outlook}
While the quality of the features as well as the general code quality could be massively improved, the project is not in a final state.
Functionality of the project could be improved even further.
Ideas include:
\begin{itemize}
    \item Automatic generation of contracts.
    \item Debugging for Dafny.
    \item Create clients for other IDE's, which connect to our Dafny language server.
\end{itemize}

Alongside the widening of the feature range, it is necessary to complete the visitor.
The visitor is the component generating an internal symbol table within the language server.
Currently, only the most important Dafny language features are supported.
For example, custom datatypes\footnote{for example \code{datatype Tree<T> = Empty | Node(left : Tree, root : T, right : Tree)}}, as used in formal programming languages, are not supported.\\

Nevertheless, the plugin is of a nice quality and is ready for deployment into the VSCode marketplace, once multi-platform support is implemented.
Thus, students can work with it and make their first steps in the Dafny programming language using our plugin.
