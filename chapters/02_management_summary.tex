\section{Management Summary}
\label{section:management_summary}
This chapter contains a brief overview of the bachelor thesis at hand.
First the technologies Dafny and Language Server protocol were explained to understand the initial solution.
This is followed by a presentation of the objectives, results achieved and future prospects.

\subsection{Dafny}
\label{section:introduction:dafny}
Dafny is formal programming language.
Our work should make it easier for Dafny developers to write programs in Dafny. \\

Dafny offers optimizations for proofs as an advantages over other programming languages.
Within the Dafny code, a developer has the option to state a so called precondition.
This is a fact that must be true at the beginning of the code.
Once the code has been executed, the developer may state a postcondition.
The postcondition is, contrary to the precondition, something that is true at the end of code.
Dafny contains an internal engine to automatically proof postconditions.
This means, given the precondition holds, the code will manipulate data only thus far, so that also the postcondition is valid.
Dafny's specialty is that it was optimized for such proofs.

\subsection{Language Server Protocol}
The language server protocol (LSP) is a specification for communication.
It regulates data exchange between an IDE and a so called language server.
This ensures a separation between IDE and the actual language support in the Language Server for a programming language (such as Dafny).
With this standard protocol, the same language server can be easily integrated by different IDEs. \\

Our Dafny Plugin was developed for the IDE "Visual Studio Code".
Visual Studio Code is for example used by programmers to write Dafny code.
The code is then transferred to the language server, using a format specified by the LSP.
The language server is now responsible to analyze the code and provide proper replies to the editor, whenever the editor requests something.
A request could for example be to show information about a variable.
The server has to provide all the necessary information to the IDE.\\

As a result of using çthe LSP we make it possible that Dafny can be supported for other IDEs with minimal effort.

\subsection{Initial Solution}
\label{section:management_summary:initialsolution}
In a previous bachelor thesis by Markus Schaden and Rafael Krucker, a LSP client-server infrastructure for Visual Studio Code was created to support Dafny \cite{ba}.
The plugin was particularly appreciated by the "HSR Correctness Lab" \cite{correctnessLab} to make coding in Dafny easier.
The language server and the Dafny backends were separated into two different components.
This had the disadvantage that they had to communicate over a proprietary interface.
This solution was not optimal, since the Dafny backend was not accessed directly.
This led to the fact that implementations of new features are very time consuming. \\

In the preceding semester project \cite{sa}, the language server was merged into the Dafny backend, so that the proprietary interface was no longer necessary.
Dafny could be accessed directly by the new software architecture.
Within this bachelor thesis, the work of the existing solution will be continued.

\subsection{Goals}
At HSR, Dafny is taught within the course \textit{Software Engineering}.
Students have to make their first steps in Dafny, using a simple tutorial.
To provide the students with the necessary support, they shall profit by common IDE features like
\begin{itemize}
    \item Syntax highlighting to make the readability more pleasant
    \item Error highlighting to correct faulty code efficiently
    \item Compile and run to run Dafny programs and check the output
    \item Auto completion to code efficiently
    \item Go to definition to be able to view definitions quickly
    \item Showin counter examples to efficiently remedy faulty pre and post conditions
\end{itemize}

On the other hand, the plugin shall also be used in a professional environment in the future.
Thus, more advanced features have to be supported.
To facilitate the future development of such features, the codebase shall be refactored to achieve a clean and maintainable state.
This especially involves a clean architectural layout with well-organized dependencies.\\

Aside a clean architecture, a major goal is the creation of a custom symbol table.
A symbol table contains information about items appearing in the Dafny code.
It shall allow very simple navigation within the Dafny code, for example to locate the declaration of a variable.
Once the symbol table is generated, navigation within the code will be very easy for the language server.
This has the advantage, that many features can be realized with almost trivial logic.
For example, the feature \textit{go to definition} just has to ask the symbol table, where the definition of the symbol is.
For example, a symbol can be a variable, function or class.

Consequently, a well implemented symbol table will also facilitate the development of further features.

\subsection{Results}
%%intro
Within this bachelor thesis, the development of the pre-existing language server and its VSCode client was continued.
Significant improvements could be achieved, which are described in this chapter.

%%Features
The following features are supported as planned:
\begin{itemize}
    \item Syntax highlighting
    \item Verification: highlighting of errors and warnings
    \item Compilation of Dafny code and execute
    \item Auto completion suggestion
    \item Go to definition
    \item Providing of Counter Examples
\end{itemize}

Additionally the following functions could be implemented thanks to the new symbol table:
\begin{itemize}
    \item Code Lens counts the usages of classes and functions and lets the developer show the usages with one click
    \item Renaming allows to rename symbols efficiently and reliably
    \item Hover information provide the developer with useful information about symbols
\end{itemize}

Dafny developers can now benefit by a plugin, which gained a lot of user experience, but also robustness compared to the initial solution.\\

%%Dev Benefits.
Aside the improvements in features, many internal aspects were also improved.
This benefits all developers who want to extend the Danfy language server to offer more functionality to the Dafny plugin. \\

A component called Dafny translation unit was completely re-visited and simplified.
The component accesses any Dafny functionality.
For example, instead of passing Dafny options as an array of strings, they are now set by
directly accessing Dafny's config class.
Any results provided by the Dafny translation unit are buffered for later re-use at compilation or to create the symbol table.
This makes the implementation significantly more performant.\\

%%ST
The targeted symbol table could be implemented for the most important Dafny language features.
Various challenges had to be accommodated, including the handling of default scopes, default classes, inheritance, external file import or variable hiding.

The symbol table opens the option to implement many more features than currently provided.
For example, the LSP offers a highlight request, marking occurrences of a symbol.
Any information required for this feature are already provided by the symbol table.
Thus, adding this feature would be very simple.

%%client
Aside changes in the server, the VSCode client is now as lightweight as possible.
This makes the adaption to other IDE's very simple.
The server is now able to create a symbol table containing any information required for the langue analysis features.
Pre-existing features and algorithms were improved to gain more reliability and a better user experience.

\subsection{Outlook}
While the quality of the features, as well as the general code quality could be massively improved, the project is not in a final state.
Functionality of the project could be improved even further.
Ideas include:
\begin{itemize}
    \item Automatic generation of contracts
    \item Debugging for Dafny
    \item Create clients for other IDE's
\end{itemize}

Aside the widening of the feature range, it is definitely necessary to complete the visitor.
The visitor is the component generating an internal symbol table within the language server.
Currently, only the most important Dafny language features are supported.
This was due to the limited time frame of the bachelor thesis.
For example, custom datatypes\footnote{for example \code{datatype Tree<T> = Empty | Node(left : Tree, root : T, right : Tree)}}, as used in formal programming languages, are not supported.\\

Nevertheless, the plugin is of a nice quality and is already deployed into the VSCode market place.
Thus, students can work with it and make their first steps in the Dafny programming language using our Dafny plugin.
