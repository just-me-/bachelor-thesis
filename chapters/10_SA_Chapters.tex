\subsection{Contend References from the Pre-existing Term Project}
The following sections originated in the pre-existing semester thesis "Dafny Server Redesign"\cite{sa} and were reincluded in this document for the sense of a comprehensive documentation.
Minor changes, such as typos or the adjustment of the preceding project, are not mentioned separately.\\

Dafny and the language server protocol are fundamental technologies of this project.
Since the external expert, as well as the third reader were not involved in the development of the prototype, the analysis of those technologies done in the preceding thesis were taken to this document.

\begin{itemize}
    \item Chapter \ref{section:abstract} \nameref{section:abstract}
        \begin{itemize}
            \item Paragraph 1 about Dafny
            \item Paragraph 2 about the language server protocol communication
            \item  \intnote{todo bei der einleitung / management summary / abstract. alles Dafny + LSP? und sonst noch: alle aussagen brauchen quellen todo}
        \end{itemize}
    \item Chapter \ref{section:introduction} \nameref{section:introduction}
        \begin{itemize}
            \item Chapter \ref{section:introduction:dafny} except for the paragraph about lemmas
            \item Chapter \ref{section:introduction:initialsolution}, partially
        \end{itemize}

    \item Chapter \ref{section:analysis} \nameref{section:analysis}
        \begin{itemize}
            \item Chapter \ref{section:analysis:lsp}
            \item Chapter \ref{section:analysis:omnisharp}, reworked, better example
            \item Chapter \ref{section:analysis:features}, adoption of the starting positions from the results. How the improvements are to be implemented is new.
        \end{itemize}

    \item Chapter \ref{section:results} \nameref{section:results}
        \begin{itemize}
            \item Chapter \ref{section:result:syntaxhighgliht}
        \end{itemize}

    \item Chapter \ref{section:project_management} \nameref{section:project_management}
        \begin{itemize}
            \item The whole chapter is structurally based on the pre-existing term project.
            Partially, sentence fragments from general introductions were taken over.
            The data as well as the conclusions from it are however new.
        \end{itemize}


\end{itemize}
