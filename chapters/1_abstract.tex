\section{Abstract}
\label{section:abstract}

\textbf{Initial Situation}\\
Dafny is a formal programming language to proof a program's correctness with preconditions, postconditions, loop invariants and loop variants.
In a preceding bachelor thesis, a plugin for Visual Studio Code had been implemented to access Dafny-specific static analysis features.
For example, if Dafny cannot prove a postcondition, the code will be highlighted and a counter example is shown.
Furthermore, it provides access to code compilation, auto completion suggestions and various automated refactorings.\\

The plugin communicates with a language server, using Microsoft's language server protocol, which standardizes communication between an integrated development environment (IDE) and a language server. 
The language server itself used to access the Dafny library, which features the backend of the Dafny language analysis, through a proprietary JSON-interface.
In a preceding term project, the language server was integrated into the Dafny backend to render the JSON-interface obsolete.\\

\textbf{Objective}\\
This bachelor thesis is a direct continuation of the preceding term project.
It has two major goals:
\begin{itemize}
    \item Improvement of previously implemented features in usability, stability and reliability.
    \item Implementation of a symbol table to facilitate the development of navigational features like rename or auto completion.
\end{itemize}
The symbol table is required to contain information about each name segment in the code.
It should allow direct access to a name segment's declaration, information about its scope and usage statistic.\\

\textbf{Result}\\
Using the visitor pattern, the Dafny abstract syntax tree (AST) is visited to create the symbol table.
Every symbol contains information about its parent, its children and its declaration.
This makes navigation within the symbol hierarchy  very simple.
The features goto definition, rename, code lens, hover information and auto completion can directly benefit by the symbol table and are no longer required to contain much logic.
This made the language server more robust.
The offerings of the symbol table can be used for future extensions, such as auto formatting Dafny code depending on the depth of scope.\\


Aside features based on the symbol table, pre-existing functionality was revisited as well:
\begin{itemize}
    \item The verification and compilation processes were simplified by creating a dedicated Dafny translation unit.
    \item Its results are buffered for efficient access.
    \item Unlike in prior versions of the plugin, warnings are now reported to the end-user as well.
    \item Counter examples are displayed in a simpler matter.
    \item By hovering over a symbol, the user receives basic information, such as the symbol type.
    \item All features will now also work across multiple files and namespaces.
\end{itemize}