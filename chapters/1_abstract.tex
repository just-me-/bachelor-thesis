\section{Abstract}
Dafny is a formal programming language to proof a program's correctness with preconditions, postconditions, loop invariants and loop variants. In a preceding bachelor thesis, a plugin for Visual Studio Code had been implemented to access Dafny-specific static analysis features. For example, if Dafny cannot prove a postcondition, the code will be highlighted and a counter example is shown. Furthermore, it provides access to code compilation, auto completion suggestions and various automated refactorings.\\

The plugin communicates with a language server, using Microsoft's language server protocol, which standardizes communication between an integrated development environment (IDE) and a language server.  The language server itself used to access the Dafny library, which features the backend of the Dafny language analysis, through a proprietary JSON-interface. In a preceding semester project, the language server was integrated into the Dafny backend to make the JSON-interface obsolete.\\

This bachelor thesis is a direct continuation of the preceding term project. It had two major goals:
\begin{itemize}
    \item Improvement of previously implemented features in usability, stability and reliability.
    \item Implementation of a symbol table.
\end{itemize}
The symbol table was required to contain information about each name segment in the code. It should allow direct access to a name segment's declaration, information about its scope and a small usage statistic.

Using the visitor pattern, the Dafny abstract syntax tree is visited to generate the symbol table. After its generation, the symbol table can be navigated from top to bottom - for example to search for a certain symbol - or from bottom to top - for example to locate all available declared symbols in a scope. Every symbol contains information about its parent, its children and its declaration. Thus, the features goto defitnion, rename, code lens and auto completion were very simple to implement.\\

Aside features based on the simbol table, preexisting functionality was revisisted as well. The verification and compilation processes were simplified by creating a dedicated translation unit. Its results are buffered for efficient access. Unlike prior versions, warnings are now designated as well. Counter examples are displayed in a simpler matter. By hovering over a symbol, the user receives basic information, such as the symbol type. All features will now also work accross multiple files and namespaces.